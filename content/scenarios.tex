\chapter{Rendszertesztek és bemutató szcenáriók}

\section{Cél, módszertan, mérőszámok}
A tesztelés célja annak igazolása, hogy a rendszer komponensei (ESP szimulátorok, vezérlő szerver, megszakító szimulátorok, Prometheus, Grafana, Dev Panel) a specifikációnak megfelelően működnek. A vizsgálat során \emph{idősoros} bemeneteket (\texttt{esp\{1..3\}\_schedule.txt}) és küszöb fájlt (\texttt{thresholds.txt}) használunk; a futtatási állapotot a \texttt{sim\_control.txt} határozza meg. A kontrollciklus periódusa \(T_c=3\,\mathrm{s}\).

Mérőszámok és ellenőrzési pontok:
\begin{itemize}
  \item \textbf{Összáram} (\texttt{sum\_current\_amps}) és \textbf{per-SIM tényleges áram} (\texttt{effective}) a vezérlő \texttt{/status} végpontján és az \texttt{output.txt}-ben.
  \item \textbf{Korlátok (cap)}: az allokáció (max--min fair) eredményei.
  \item \textbf{Küszöbök}: \texttt{ALLOC\_MAX\_TOTAL}, \texttt{BREAKER\_MAX\_TOTAL}, \texttt{BREAKER\_MIN\_TOTAL}.
  \item \textbf{Megszakító állapot}: \texttt{on/off} (hiszterézis).
\end{itemize}

A bizonyítékokat három helyen ellenőrizzük: (i) Dev Panel élő mini-grafikon, (ii) \texttt{/status} JSON pillanatkép, (iii) \texttt{output.txt} idősoros napló (egy sor/ciklus).

\section{Tesztkörnyezet}
Docker Compose alapú környezet: három ESP szimulátor, két megszakító szimulátor, a \texttt{control\_server} (Flask), Prometheus és Grafana, valamint a Dev Panel. A fájlalapú bemenetek és kimenetek a \texttt{./data} kötetben találhatók (UTF--8, soronként értelmezve, \texttt{\#} = megjegyzés).

\section{Bemenetek és állapot}
\begin{itemize}
  \item \texttt{thresholds.txt}: \texttt{BREAKER\_MAX\_TOTAL}, \texttt{BREAKER\_MIN\_TOTAL}, \texttt{ALLOC\_MAX\_TOTAL}.
  \item \texttt{esp1\_schedule.txt}, \texttt{esp2\_schedule.txt}, \texttt{esp3\_schedule.txt}: \texttt{seconds\ \ \ amps} párok, lépcsős érvényességgel.
  \item \texttt{sim\_control.txt}: \texttt{RUNNING}/\texttt{STOPPED}; alapértelmezés: \texttt{STOPPED}.
\end{itemize}

\section{Várt viselkedés (rövid)}
\begin{enumerate}
  \item Ha \(\sum d_i \le \texttt{ALLOC\_MAX\_TOTAL}\)\,: \emph{nincs korlát} (cap \(\to\) nagy érték), \(\mathrm{effective}_i=d_i\).
  \item Ha \(\sum d_i > \texttt{ALLOC\_MAX\_TOTAL}\)\,: \emph{max--min fair} (water-filling) elosztás: \(a_i=\min\{d_i,\lambda\}\), \(\sum_i a_i = \texttt{ALLOC\_MAX\_TOTAL}\).
  \item Megszakító: ha \(\mathrm{sum} \ge \texttt{BREAKER\_MAX\_TOTAL}\)\,→ \texttt{off}; ha \(\mathrm{sum} \le \texttt{BREAKER\_MIN\_TOTAL}\)\,→ \texttt{on}.
  \item \texttt{STOPPED} állapotban a virtuális idő nem halad, a vezérlő nem küld új cap-et és nem kapcsolgat megszakítót.
\end{enumerate}

\section{Szcenáriók és elfogadási kritériumok}
Az alábbi szakaszok egységes szerkezetben szerepelnek: \emph{Előkészítés} (főbb beállítások), \emph{Lépések}, \emph{Várt eredmény}, \emph{Siker kritérium}. A konkrét bemeneti fájlok a tesztcsomagban mellékeltek.

\subsection{1.\ Smoke test: Start/Stop/Reset/Clear}
\textbf{Előkészítés:} \texttt{BREAKER\_MAX\_TOTAL=12}, \texttt{BREAKER\_MIN\_TOTAL=2}, \texttt{ALLOC\_MAX\_TOTAL=30}; ESP1=1.0 A, ESP2=1.5 A, ESP3=0.5 A.\\
\textbf{Lépések:} STOP \(\to\) Reset \(t{=}0\) \(\to\) START.\\
\textbf{Várt eredmény:} nincs korlát (cap \(\approx\) INF), Sum \(\approx 3.0\) A, megszakítók \texttt{on}.\\
\textbf{Siker:} élő grafikonon vízszintes \(\approx 3\) A; \texttt{/status} tükrözi; az \texttt{output.txt} 3 s-enként bővül.

\subsection{2.\ Alulterhelés: nincs korlátozás}
\textbf{Előkészítés:} \texttt{ALLOC\_MAX\_TOTAL=6}, \texttt{BREAKER\_MAX\_TOTAL=12}, \texttt{BREAKER\_MIN\_TOTAL=2}; ESP1=2.0 A, ESP2=1.5 A, ESP3=0.5 A (Sum=4.0 A).\\
\textbf{Lépések:} START, várakozás \(\sim\) 2 ciklus.\\
\textbf{Várt eredmény:} \(\mathrm{effective}_i = d_i\), cap \(\approx\) INF.\\
\textbf{Siker:} \texttt{output.txt}-ben minden SIM-nél \texttt{cap} nagy (``nincs korlát''); grafikonon Sum \(\approx 4\) A.

\subsection{3.\ Túlterhelés, azonos igények: fair 3/3/3}
\textbf{Előkészítés:} \texttt{ALLOC\_MAX\_TOTAL=9}, \texttt{BREAKER\_MAX\_TOTAL=12}, \texttt{BREAKER\_MIN\_TOTAL=2}; ESP1=50 A, ESP2=50 A, ESP3=50 A.\\
\textbf{Lépések:} START, várakozás \(\sim\) 2--3 ciklus.\\
\textbf{Várt eredmény:} \(\lambda=9/3=3\) A \(\Rightarrow\) \(\mathrm{effective}=[3,3,3]\), Sum \(=9\) A.\\
\textbf{Siker:} grafikonon három azonos szint \(\approx 3\) A; \texttt{/status} és \texttt{output.txt} szerint \texttt{cap}\(=3\) A mindháromnál.

\subsection{4.\ Dinamikus újraelosztás: a nagy felhasználó kap teret}
\textbf{Előkészítés:} \texttt{ALLOC\_MAX\_TOTAL=90}, \texttt{BREAKER\_MAX\_TOTAL=120}, \texttt{BREAKER\_MIN\_TOTAL=10}. Menetrendek: \\
ESP1: \(0\) s \(\to 50\) A, \(60\) s \(\to 10\) A; \;
ESP2: \(0\) s \(\to 50\) A, \(60\) s \(\to 10\) A; \;
ESP3: \(0\) s \(\to 50\) A, \(60\) s \(\to 100\) A.\\
\textbf{Lépések:} Reset \(t{=}0\), START, megfigyelés \(0..80\) s.\\
\textbf{Várt eredmény:} \(0..60\) s: \([30,30,30]\), Sum \(=90\) A; \(60+\) s: \([10,10,70]\), Sum \(=90\) A.\\
\textbf{Siker:} a grafikon két platót mutat: előbb \(30\)/\(30\)/\(30\), majd \(10\)/\(10\)/\(70\).

\subsection{5.\ Megszakító hiszterézis}
\textbf{Előkészítés:} \texttt{ALLOC\_MAX\_TOTAL=50}, \texttt{BREAKER\_MAX\_TOTAL=6}, \texttt{BREAKER\_MIN\_TOTAL=3}. Menetrendek: \\
ESP1: \(0\) s \(\to 2.0\) A, \(40\) s \(\to 0.5\) A; \;
ESP2: \(0\) s \(\to 5.0\) A, \(40\) s \(\to 0.5\) A; \;
ESP3: \(0\) s \(\to 0.0\) A.\\
\textbf{Lépések:} Reset \(t{=}0\), START, megfigyelés \(0..60\) s.\\
\textbf{Várt eredmény:} \(0..40\) s Sum \(=7\) A \(\Rightarrow\) \texttt{off}; \(40+\) s Sum \(=1\) A \(\Rightarrow\) \texttt{on}.\\
\textbf{Siker:} \texttt{breakers=\dots:off,\dots:off} \(\to\) \texttt{on,\ on} váltás az \texttt{output.txt}-ben és \texttt{/status}-ban.

\subsection{6.\ STOPPED invariánsok}
\textbf{Előkészítés:} induljunk a 3.\ szcenárió állapotából (3/3/3 cap).\\
\textbf{Lépések:} \texttt{STOPPED} módba váltás; módosítsuk \texttt{ALLOC\_MAX\_TOTAL=6}-ra, ESP1 menetrendjét \(1\) A-ra; várjunk \(\sim\) 2 ciklust.\\
\textbf{Várt eredmény:} a cap-ek és a megszakítóállapot \textbf{nem} változik (\texttt{STOPPED} alatt nem történik beavatkozás).\\
\textbf{Siker:} \texttt{/status.sim\_state=STOPPED}; a \texttt{cap} és a \texttt{breakers} mezők változatlanok.

\section{Napló és bizonyítékok rögzítése}
Az \texttt{output.txt} minden ciklusban egy sort ír \texttt{kulcs=érték} párok formájában, például:
\begin{verbatim}
timestamp=... sim_state=RUNNING sum_current_amps=9.0 alloc_max_total_amps=9.0
sims=esp1:raw=50,effective=3,cap=3|esp2:raw=50,effective=3,cap=3|esp3:raw=50,effective=3,cap=3
breakers=brk1:on,brk2:on
\end{verbatim}
A szcenáriók elfogadásához javasolt minden tesztnél egy \emph{háromrészes} bizonyítékszettet elmenteni: (i) 1--2 reprezentatív naplósor, (ii) \texttt{/status} pillanatkép (JSON), (iii) a Dev Panel élő grafikonjának képernyőképe.

\section{Eredmények összegyűjtése (űrlap)}
Az alábbi táblázat a jegyzőkönyv részeként használható; soronként egy szcenárió:
\begin{center}
\begin{tabular}{p{3.2cm} p{5.6cm} p{5.6cm}}
\hline
\textbf{Szcenárió} & \textbf{Várt eredmény (rövid)} & \textbf{Megfigyelés / linkek} \\
\hline
1.\ Smoke test & Sum \(\approx 3\) A, cap=INF, breakers on & \\
2.\ No caps & Sum \(\approx 4\) A, cap=INF & \\
3.\ Fair split & \([3,3,3]\), Sum \(=9\) A & \\
4.\ Újraelosztás & \([30,30,30] \to [10,10,70]\), Sum \(=90\) A & \\
5.\ Hiszterézis & off \( \to \) on 40 s körül & \\
6.\ STOPPED & cap/Breaker invariáns & \\
\hline
\end{tabular}
\end{center}

\section*{Összegzés}
A tesztek igazolják, hogy (i) az allokáció a max--min fair elvet követi, (ii) a megszakító hiszterézise a küszöbökhöz kötötten működik, (iii) a \texttt{STOPPED} állapot invariánsai teljesülnek, és (iv) a rendszer minden ciklusban önleíró idősoros naplót készít. Ezek együtt biztosítják a funkcionális helyességet és a bemutatók során elvárt, átlátható viselkedést.
