\chapter{Rendszertesztek és bemutató szcenáriók}

\section{Tesztek megvalósítása}
A cél itt annak igazolása volt, hogy a rendszer komponensei megfelelően működnek. 
A vizsgálat során \emph{idősoros} bemeneti és kimeneti fájlt (\texttt{thresholds.txt}) használtam. 
Ebben az esetben a kontrollciklus periódusa \(T_c=3\,\mathrm{s}\). \newline

\noindent \textbf{Mérőszámok és ellenőrzési pontok:}
\begin{itemize}
    \item \textbf{Mérőnkénti tényleges áram} (\texttt{effective}) ez nem az igényelt hanem a ténylegesen megkapott 
    áramerősség, a vezérlő \texttt{/status} végpontján és az \texttt{output.txt}-ben.
    \item \textbf{Összáram} A Mérőnkénti tényleges áramok összege (\texttt{sum\_current\_amps})
    \item \textbf{Korlátok (cap)}: az allokált teljesítmény a végpontokon (max--min fair) eredményei.
    \item \textbf{Küszöbök}:
    \begin{itemize}
        \item \texttt{ALLOC\_MAX\_TOTAL} - Ez a teljes teljesítmény keret, amit a vezérlő ki tud osztani,
        a kiosztott áramok összege legfeljebb ennyi lehet.
        \item \texttt{BREAKER\_MAX\_TOTAL} - A védelem kapcsolásának küszöbe, ha az összáram 
        meghaladja ezt az értéket, a megszakítók lekapcsolnak (OFF).
        \item \texttt{BREAKER\_MIN\_TOTAL} - A védelem automatikus visszakapcsolásának küszöbe,
        csak akkor kapcsol vissza (ON) a megszakító, ha az összáram ez alá csökken.
    \end{itemize}
    \item \textbf{Megszakító állapot}: Itt csak \texttt{on/off} értéket figyelünk a védelmet ellátó megszakítókon.
\end{itemize}

\noindent Ezeket \texttt{output.txt} idősoros naplóban ellenőriztem, itt volt a legegyszerűbb, 
mert itt egy sor egy ciklus.

\section{Bemenetek és állapot}
\begin{itemize}
    \item \texttt{thresholds.txt}:
    \begin{itemize}
        \item \texttt{BREAKER\_MAX\_TOTAL}
        \item \texttt{BREAKER\_MIN\_TOTAL}
        \item \texttt{ALLOC\_MAX\_TOTAL}
    \end{itemize}
    Ezek fentebb említett módon kerülnek használatra.
    \item idő - áramerősség párokat tartalmazzó menetrendek minden végponthoz:
    \begin{itemize}
        \item \texttt{esp1\_schedule.txt}
        \item \texttt{esp2\_schedule.txt}
        \item \texttt{esp3\_schedule.txt}
    \end{itemize}
    \item \texttt{sim\_control.txt}:
    \begin{itemize}
        \item RUNNING a szimulációhoz használt merők belső órája megy
        \item STOPPED a szimulációhoz használt mérők belső órája megáll
    \end{itemize}
    alapértelmezés: STOPPED.
\end{itemize}

\section{Várt viselkedés}
\begin{enumerate}
  \item Ha \(\sum_i d_i \le \texttt{ALLOC\_MAX\_TOTAL}\)\,: \emph{nincs korlát}, 
  ezért \(\mathrm{effective}_i=d_i\) minden mérőre és az összáram egyszerűen \(\sum_i d_i\).
  Ilyenkor a vezérlő nem „oszt újra”, a kiosztás megegyezik az igényekkel 
  és a megszakító-logika csak akkor lép működésbe, ha az összáram véletlenül mégis átlépi a védelmi küszöböt.

  \item Ha \(\sum_i d_i > \texttt{ALLOC\_MAX\_TOTAL}\)\,: \emph{max--min fair} elosztás lép életbe,
  vagyis egy \(\lambda\) szintet keresünk úgy, hogy \(a_i=\min\{d_i,\lambda\}\) 
  és \(\sum_i a_i=\texttt{ALLOC\_MAX\_TOTAL}\).
  Azok a mérők, amelyek igénye \(d_i \le \lambda\), teljes igényüket megkapják, 
  a nagyobb igényűek pedig \(\lambda\)-nál „levágódnak”, a vezérlő ezt 3 s-onként újraszámolja, így 
  ha szabadul fel kapacitás ez automatikusan átcsoportosul.

  \item Megszakító: ha az \emph{összáram} \(\mathrm{sum} \ge \texttt{BREAKER\_MAX\_TOTAL}\), 
  a megszakító kikapcsol (védelmi leoldás)
  és csak akkor kapcsol vissza, ha \(\mathrm{sum} \le \texttt{BREAKER\_MIN\_TOTAL}\).

  \item \texttt{STOPPED} állapotban a virtuális idő nem halad, a vezérlő nem küld új korlátokat és 
  nem ad megszakító-parancsokat, ilyenkor a bemeneti fájlok szabadon szerkeszthetők,
  és a következő \texttt{RUNNING} ciklus kezdetekor az új konfiguráció lép életbe, ha ez be van kattintva 
  időnullázással és naplóürítéssel.
\end{enumerate}


\section{Szcenáriók és elfogadási kritériumok}

\subsection{Alaptesztek: Start/Stop/Reset/Clear}
\textbf{Bemenetek:}
\begin{itemize}
    \item \texttt{BREAKER\_MAX\_TOTAL=12}
    \item \texttt{BREAKER\_MIN\_TOTAL=2}
    \item \texttt{ALLOC\_MAX\_TOTAL=30}
    \item ESP1=1{,}0~A
    \item ESP2=1{,}5~A
    \item ESP3=0{,}5~A
\end{itemize}

\textbf{Miért ez a beállítás?} Az igények összege \(1{,}0+1{,}5+0{,}5=3{,}0\)~A \(\ll \texttt{ALLOC\_MAX\_TOTAL}\), 
ezért nem várható korlátozás: \(\mathrm{effective}_i=d_i\). \newline

\vspace{2pt}
\noindent\textbf{Lépések és jelentésük:}
\begin{enumerate}
    \item \textbf{STOP} — a \texttt{sim\_control.txt} \texttt{STOPPED}-ra állítása, 
    a vezérlő nem küld új korlátotokat 
    és nem is kapcsol megszakítót.
    \item \textbf{Reset \(t{=}0\)} — minden szimulátor idejét nullázzuk, 
    a szimuláció elejéről kezdünk.
    \item \textbf{START} — a vezérlő elindul, a következő ciklusban kiírja az állapotot 
    és beállítja a korlátokat, de ebben az estben nem kell.
    \item \emph{(Opcionális) Clear \texttt{output.txt}} törölhető a napló amennyiben 
    tiszta fájlt szeretne valaki látni.
\end{enumerate}

\textbf{Várt rendszerállapot}
\begin{itemize}
    \item \emph{Nincs korlát}: \(\mathrm{effective}_{1,2,3}=\{1{,}0;1{,}5;0{,}5\}\)~A, 
    a korlátokat nagy értékek (\(\sim 10^9\)) jelölik.
    \item \emph{Összáram}: \(\texttt{sum\_current\_amps}\approx 3{,}0\)~A stabilan, 
    ez így vízszintes vonal az élő grafikonon.
    \item \emph{Megszakító}: bekapcsolt állapotban van, 
    mert \(3{,}0<\texttt{BREAKER\_MAX\_TOTAL}=12\) és \(3{,}0>\texttt{BREAKER\_MIN\_TOTAL}=2\).
\end{itemize}

\textbf{Hol ellenőrizhető?}
\begin{itemize}
    \item \texttt{output.txt}: 3~s-onként új sor, pl.:
    \begin{lstlisting}
    sim_state=RUNNING sum_current_amps=3.0
    sims=esp1:raw=1.0,effective=1.0,cap=1e9|esp2:raw=1.5,
    effective=1.5,cap=1e9|esp3:raw=0.5,effective=0.5,cap=1e9
    breakers=brk1:on,brk2:on
    \end{lstlisting}
\end{itemize}

\textbf{Siker kritérium:} A grafikon ~3~A változatlan értékű görbét mutat 
az \texttt{output.txt-ből} ugyanezt tudjuk kiolvasni, 
hogy \(\mathrm{effective}_i=d_i\), korlátok nincsenek érvényben, a megszakítók 
be vannak kapcsolva ez 1-2 ciklus (3-6~s) után stabilan látszik.


\begin{figure}[H]
    \centering
    \includegraphics[width=1\textwidth]{figures/alaptesztek_1.png}
    \caption{Alaptesztek}
    \label{fig:alaptesztek}
\end{figure}

\subsection{Alulterhelés: nincs korlátozás}
\textbf{Bemenetek:}
\begin{itemize}
    \item \texttt{ALLOC\_MAX\_TOTAL=6}
    \item \texttt{BREAKER\_MAX\_TOTAL=12}
    \item \texttt{BREAKER\_MIN\_TOTAL=2}
    \item ESP1=2{,}0~A
    \item ESP2=1{,}5~A
    \item ESP3=0{,}5~A
\end{itemize}

\textbf{Miért ez a beállítás?} Az igények összege 
\(2{,}0+1{,}5+0{,}5=4{,}0\)~A \(\le \texttt{ALLOC\_MAX\_TOTAL}=6\), ezért \emph{nem} indul korlátozás 
(max--min fair kiosztásra nincs szükség), így \(\mathrm{effective}_i=d_i\). A 4{,}0~A az \texttt{BREAKER\_MIN}
 és \texttt{BREAKER\_MAX} között van, ezért a megszakítók \emph{bekapcsolt} állapotban maradnak. \newline

\vspace{2pt}
\noindent\textbf{Lépések és jelentésük:}
\begin{enumerate}
    \item \textbf{START} — a vezérlő elindul, és kiírja az állapotot, 
    mivel \(\sum d_i \le \texttt{ALLOC\_MAX}\), a korlátokat nem kell érvényesíteni.
    \item \textbf{Várakozás \(\sim\) 2 ciklus} — 6--7~s múlva a naplóban stabilan láthatóak a 
    beállítások.
\end{enumerate}

\textbf{Várt rendszerállapot}
\begin{itemize}
    \item \emph{Nincs korlát}: \(\mathrm{effective}_{1,2,3}=\{2{,}0;1{,}5;0{,}5\}\)~A, 
    a korlátok nagy értékekkel (\(\sim 10^9\)) jelzik a „nincs limit” állapotot.
    \item \emph{Összáram}: \(\texttt{sum\_current\_amps}\approx 4{,}0\)~A \(\Rightarrow\) 
    vízszintes vonal az élő grafikonon.
    \item \emph{Megszakító}: bekapcsolva, mert 
    \(4{,}0<\texttt{BREAKER\_MAX\_TOTAL}=12\) és \(4{,}0>\texttt{BREAKER\_MIN\_TOTAL}=2\).
\end{itemize}

\textbf{Hol ellenőrizhető?}
\begin{itemize}
    \item \texttt{output.txt}: 3~s-onként új sor, pl.:
\begin{lstlisting}
sim_state=RUNNING sum_current_amps=4.0
sims=esp1:raw=2.0,effective=2.0,cap=1e9|esp2:raw=1.5,
effective=1.5,cap=1e9|esp3:raw=0.5,effective=0.5,cap=1e9
breakers=brk1:on,brk2:on
\end{lstlisting}
\end{itemize}

\textbf{Siker kritérium:} az élő grafikon ~4~A vízszintes görbét mutat, 
az \texttt{output.txt} egyezően jelzi, hogy \(\mathrm{effective}_i=d_i\), korlát nincs érvényben, 
a megszakítók be vannak kapcsolva mindez 1--2 ciklus (3--6~s) után stabilan látszik.


\begin{figure}[H]
    \centering
    \includegraphics[width=1\textwidth]{figures/undercap_2.png}
    \caption{Alulterhelt eset}
    \label{fig:undercap}
\end{figure}

\subsection{Túlterhelés, azonos igények: fair 3/3/3 allokáció}
\textbf{Bemenetek:}
\begin{itemize}
    \item \texttt{ALLOC\_MAX\_TOTAL=9}
    \item \texttt{BREAKER\_MAX\_TOTAL=12}
    \item \texttt{BREAKER\_MIN\_TOTAL=2}
    \item ESP1=50~A
    \item ESP2=50~A
    \item ESP3=50~A
\end{itemize}

\textbf{Miért ez a beállítás?} Az igények összege 
\(50+50+50=150\)~A \(\gg \texttt{ALLOC\_MAX\_TOTAL}=9\), ezért a max--min fair elosztás: egy közös \(\lambda\) 
szintet keresünk úgy, hogy \(a_i=\min\{d_i,\lambda\}\) és \(\sum a_i=9\). 
Azonos igények mellett \(\lambda=9/3=3\)~A, tehát minden mérő 3~A-t kap.

\vspace{2pt}
\noindent\textbf{Lépések és jelentésük:}
\begin{enumerate}
    \item \textbf{START} — a vezérlő elindul, kiszámítja \(\lambda\)-t és beállítja a korlátokat.
    \item \textbf{Várakozás \(\sim\) 2 ciklus} — 6~s alatt a napló stabilan tükrözi a 3/3/3 kiosztást.
\end{enumerate}

\textbf{Várt rendszerállapot}
\begin{itemize}
    \item \emph{Korlátok}: mindhárom mérőnél \(\texttt{cap}=3\)~A; \(\mathrm{effective}_{1,2,3}=\{3,3,3\}\)~A.
    \item \emph{Összáram}: \(\texttt{sum\_current\_amps}=3+3+3=9\)~A \(\Rightarrow\) 
    egyenes vonal a grafikonon.
    \item \emph{Megszakító}: bekapcsolva, mert \(9<\texttt{BREAKER\_MAX\_TOTAL}=12\) és \(9>\texttt{BREAKER\_MIN\_TOTAL}=2\).
\end{itemize}

\textbf{Hol ellenőrizhető?}
\begin{itemize}
    \item \texttt{output.txt}: 3~s-onként új sor, pl.:
\begin{lstlisting}
sim_state=RUNNING sum_current_amps=9.0
sims=esp1:raw=50.0,effective=3.0,cap=3.0|esp2:raw=50.0,
effective=3.0,cap=3.0|esp3:raw=50.0,effective=3.0,cap=3.0
breakers=brk1:on,brk2:on
\end{lstlisting}
\end{itemize}

\textbf{Siker kritérium:} az élő grafikon három, közel azonos (3~A) szintet és 9~A összáramot mutat, 
az \texttt{output.txt} \(\texttt{cap}=3\)~A értéket jelez mindhárom mérőnél, 
a megszakítók bekapcsolva vannak mindez 2 ciklus (6~s) után stabil.


\begin{figure}[H]
    \centering
    \includegraphics[width=1\textwidth]{figures/túlterhelés_3.png}
    \caption{Túlterhelt eset}
    \label{fig:Túlterhelt}
\end{figure}

\subsection{Dinamikus újraelosztás: a nagy felhasználó kap teret}
\textbf{Bemenetek:}
\begin{itemize}
    \item \texttt{ALLOC\_MAX\_TOTAL=90}
    \item \texttt{BREAKER\_MAX\_TOTAL=120}
    \item \texttt{BREAKER\_MIN\_TOTAL=10}
    \item Menetrendek:
    \begin{itemize}
        \item ESP1: \(0\)~s \(\to 50\)~A, \(60\)~s \(\to 10\)~A
        \item ESP2: \(0\)~s \(\to 50\)~A, \(60\)~s \(\to 10\)~A
        \item ESP3: \(0\)~s \(\to 50\)~A, \(60\)~s \(\to 100\)~A
    \end{itemize}
\end{itemize}

\textbf{Miért ez a beállítás?} Az elején az igények azonosak 
(\(50,50,50\)~A), ez túl nagy a 90~A kerethez képest, ezért \emph{fair} 
elosztás lép életbe: \([30,30,30]\)~A. Egy idő után két mérő visszaesik 
10~A-ra, a harmadik 100~A-t kér; a felszabaduló 80~A-ből a keret kitöltéséhez 
a harmadik kap 70~A-t, így \([10,10,70]\)~A lesz a kiosztás. \newline

\vspace{2pt}
\noindent\textbf{Lépések és jelentésük:}
\begin{enumerate}
    \item \textbf{Reset \(t{=}0\)} — szinkron kezdet mindenki biztosan 0-ról indul.
    \item \textbf{START} — a vezérlő 3~s-os ciklusokban számolja újra az allokációt, 
    a \(t=60\)~s utáni váltás az ezt követő ciklusban fog megjelenni.
    \item \textbf{Megfigyelés \(0..80\)~s} — várjuk \(t=60\)~s-nél az újra osztást 
    a grafikonon.
\end{enumerate}

\textbf{Várt rendszerállapot}
\begin{itemize}
    \item \emph{\(0..60\)~s}: tényleges értékek \([30,30,30]\)~A, összáram \(=90\)~A.
    \item \emph{\(60+\)~s}: ténylegesek \([10,10,70]\)~A, összáram \(=90\)~A 
    (a két kicsi igény teljesül, a maradék, pedig a nagyhoz kerül).
    \item \emph{Megszakító}: végig be van kapcsolva, 
    mert \(90<\texttt{BREAKER\_MAX}=120\) és \(90>\texttt{BREAKER\_MIN}=10\).
\end{itemize}

\textbf{Hol ellenőrizhető?}
\begin{itemize}
    \item \texttt{output.txt}: 3~s-onként új sor; jellemző minták:
\begin{lstlisting}
# t=3s  (elso szakasz)
sim_state=RUNNING sum_current_amps=90.0
sims=esp1:raw=50.0,effective=30.0,cap=30.0|esp2:raw=50.0,
effective=30.0,cap=30.0|esp3:raw=50.0,effective=30.0,cap=30.0
breakers=brk1:on,brk2:on

# t=63s (masodik szakasz, igényváltozás utan)
sim_state=RUNNING sum_current_amps=90.0
sims=esp1:raw=10.0,effective=10.0,cap=10.0|esp2:raw=10.0,
effective=10.0,cap=10.0|esp3:raw=100.0,effective=70.0,cap=70.0
breakers=brk1:on,brk2:on
\end{lstlisting}
\end{itemize}

\textbf{Siker kritérium:} a grafikon két állapotba áll be: 
előbb \(30/30/30\)~A, majd a \(t=60\)~s után 
\(10/10/70\)~A, az összáram végig \(90\)~A, 
a megszakítók végig bekapcsolva maradnak.


\begin{figure}[H]
    \centering
    \includegraphics[width=1\textwidth]{figures/dinamikus újraelosztás_4_1.png}
    \caption{Dinamikus újraelosztás áramerősség}
    \label{fig:dynamic_reallocation_1}
\end{figure}

\begin{figure}[H]
    \centering
    \includegraphics[width=1\textwidth]{figures/dinamikus újraelosztás_4_2.png}
    \caption{Dinamikus újraelosztás igények}
    \label{fig:dynamic_reallocation_2}
\end{figure}

\subsection{Megszakító hiszterézis}
\textbf{Bemenetek:}
\begin{itemize}
    \item \texttt{ALLOC\_MAX\_TOTAL=50}
    \item \texttt{BREAKER\_MAX\_TOTAL=6}
    \item \texttt{BREAKER\_MIN\_TOTAL=3}
    \item Menetrendek:
    \begin{itemize}
        \item ESP1: \(0\)~s \(\to 2{,}0\)~A,\; \(40\)~s \(\to 0{,}5\)~A
        \item ESP2: \(0\)~s \(\to 5{,}0\)~A,\; \(40\)~s \(\to 0{,}5\)~A
        \item ESP3: \(0\)~s \(\to 0{,}0\)~A
    \end{itemize}
\end{itemize}

\textbf{Miért ez a beállítás?} Kezdetben az összáram 
\(2{,}0+5{,}0+0{,}0=7{,}0\)~A, ami nagyobb, mint \(\texttt{BREAKER\_MAX\_TOTAL}=6\)~A, 
ezért a megszakító \emph{leold}, \(40\)~s után a terhelések 
\(0{,}5+0{,}5+0{,}0=1{,}0\)~A-ra esnek, ami kissebb, mint
\(\texttt{BREAKER\_MIN\_TOTAL}=3\)~A, így a megszakító \emph{visszakapcsol}. 
Az \texttt{ALLOC\_MAX\_TOTAL}=50~A bőven a terhelések fölött van, 
ezért \emph{allokációs korlátozás nem várható}. \newline

\vspace{2pt}
\noindent\textbf{Lépések és jelentésük:}
\begin{enumerate}
    \item \textbf{Reset \(t{=}0\)} — szinkron indulás.
    \item \textbf{START} — a vezérlő 3~s-os ciklusokban értékeli ki az adatokat. 
    Az első ciklusban a \(\ge 6\)~A miatt ki van kapcsolva, a \(40\)~s utáni 
    első ciklusban a \(\le 3\)~A miatt bekapcsol.
    \item \textbf{Megfigyelés \(0..60\)~s} — a naplóban egy 
    \texttt{off} \(\to\) \texttt{on} átmenet látszik \(40\)~s-nél.
\end{enumerate}

\textbf{Várt rendszerállapot}
\begin{itemize}
    \item \emph{\(0..40\)~s}: összáram \(\approx 7{,}0\)~A \(\Rightarrow\) 
    megszakító kikapcsolva.
    \item \emph{\(40+\)~s}: összáram \(\approx 1{,}0\)~A \(\Rightarrow\) 
    megszakító bekapcsolva.
    \item \emph{Allokáció}: nincs korlát (nagy szám), 
    mert \(\sum d_i \ll \texttt{ALLOC\_MAX\_TOTAL}\).
\end{itemize}

\textbf{Hol ellenőrizhető?}
\begin{itemize}
    \item \texttt{output.txt}: 3~s-onként új sor; tipikus minták:
\begin{lstlisting}
# t=3s  (kezdeti szakasz)
sim_state=RUNNING sum_current_amps=7.0
sims=esp1:raw=2.0,effective=2.0,cap=1e9|esp2:raw=5.0,
effective=5.0,cap=1e9|esp3:raw=0.0,effective=0.0,cap=1e9
breakers=brk1:off,brk2:off

# t=42s (menetrendváltás utáni első ciklus)
sim_state=RUNNING sum_current_amps=1.0
sims=esp1:raw=0.5,effective=0.5,cap=1e9|esp2:raw=0.5,
effective=0.5,cap=1e9|esp3:raw=0.0,effective=0.0,cap=1e9
breakers=brk1:on,brk2:on
\end{lstlisting}
\end{itemize}

\textbf{Siker kritérium:} az élő grafikonon \(0..40\)~s között 
\(\sim 7\)~A körüli összáram mellett kikapcsolt állapot látszik, 
\(40\)~s után \(\sim 1\)~A mellett bekapcsolt, az \texttt{output.txt} 
ezt a \texttt{off} \(\to\) \texttt{on} váltást 2-3 
cikluson belül egyértelműen tükrözi.


\begin{figure}[H]
    \centering
    \includegraphics[width=1\textwidth]{figures/megszakító_5_1.png}
    \caption{Megszakító áramerősség}
    \label{fig:megszakító_1}
\end{figure}

\begin{figure}[H]
    \centering
    \includegraphics[width=1\textwidth]{figures/megszakító_5_2.png}
    \caption{Megszakító állapotok}
    \label{fig:megszakító_2}
\end{figure}

\subsection{Leállított mód (STOPPED)}
\textbf{Bemenetek:}
\begin{itemize}
    \item Kiindulás: a 3.\ szcenárió stabil állapota 
    \begin{itemize}
        \item \(\texttt{ALLOC\_MAX\_TOTAL}=9\)
        \item \(\texttt{BREAKER\_MAX\_TOTAL}=12\)
        \item \(\texttt{BREAKER\_MIN\_TOTAL}=2\)
        \item \(\mathrm{effective}=[3,3,3]\)~A
        \item \(\texttt{sum}=9\)~A, megszakítók \texttt{on}
    \end{itemize}
    \item Módosítások STOPPED állapot alatt \emph{csak fájlba írva}:
    \begin{itemize}
        \item \(\texttt{ALLOC\_MAX\_TOTAL}=6\)
        \item ESP1 menetrend \(1{,}0\)~A
    \end{itemize}
\end{itemize}

\textbf{Miért ez a beállítás?} A cél annak igazolása, 
hogy \texttt{STOPPED} módban a vezérlő \emph{nem avatkozik be}: 
nem számol ad új korlátokat, nem kapcsol megszakítót, a virtuális idő nem halad 
és a napló is megáll.

\vspace{2pt}
\noindent\textbf{Lépések és jelentésük:}
\begin{enumerate}
    \item \textbf{STOPPED} — a \texttt{sim\_control.txt} \texttt{STOPPED}-re áll, 
    a vezérlési ciklus megáll, nincs új allokáció vagy megszakítóparancs.
    \item \textbf{Konfiguráció módosítása} — \(\texttt{ALLOC\_MAX\_TOTAL}=6\) 
    és ESP1 \(1{,}0\)~A változtatások \emph{mentése} (de ez még nem lép életbe).
    \item \textbf{Várakozás \(\sim\) 2 ciklusnyi időtartam} — mivel a rendszer áll, 
    nem várható új státusz- vagy naplósor.
\end{enumerate}

\textbf{Várt rendszerállapot}
\begin{itemize}
    \item \emph{Korlátok és effektív értékek}: változatlanul 
    \([3,3,3]\)~A (a leállítás \emph{előtti} állapot szerint).
    \item \emph{Megszakító}: változatlanul bekapcsolt állapotban van 
    (nem történik átkapcsolás).
    \item \emph{Idő és napló}: a virtuális idő nem halad, 
    az \texttt{output.txt} \emph{nem} bővül.
\end{itemize}

\textbf{Hol ellenőrizhető?}
\begin{itemize}
    \item \texttt{output.txt}: az utolsó \texttt{RUNNING} sor után nem jelennek meg új bejegyzések \texttt{STOPPED} alatt, pl.:
\begin{lstlisting}
# utolsó sor STOPPED előtt
sim_state=RUNNING sum_current_amps=9.0
sims=esp1:raw=50.0,effective=3.0,cap=3.0|esp2:raw=50.0,effective=3.0,cap=3.0|esp3:raw=50.0,effective=3.0,cap=3.0
breakers=brk1:on,brk2:on

# STOPPED alatt nincs új sor
\end{lstlisting}
\end{itemize}

\textbf{Siker kritérium:} \(\texttt{/status.sim\_state=STOPPED}\), a korlátok 
és megszakító értékek megegyeznek a leállítás \emph{előtti} állapottal 
és az \texttt{output.txt} nem bővül a várakozás ideje alatt.


\begin{figure}[H]
    \centering
    \includegraphics[width=1\textwidth]{figures/stop_6_1.png}
    \caption{stopped állapot áramerősség}
    \label{fig:stop_1}
\end{figure}

\begin{figure}[H]
    \centering
    \includegraphics[width=1\textwidth]{figures/stop_6_2.png}
    \caption{stopped állapot állapotok}
    \label{fig:stop_2}
\end{figure}

\begin{figure}[H]
    \centering
    \includegraphics[width=1\textwidth]{figures/stop_6_3.png}
    \caption{stopped állapot maximális áramok}
    \label{fig:stop_3}
\end{figure}

\section*{Összegzés}
A tesztek ellenőrzik, hogy (i) az allokáció a max-min fair elvet követi-e, (ii) 
a megszakító hiszterézise a küszöbértékekhez képest működik-e, (iii) a \texttt{STOPPED}
állapot működik-e, és (iv) a rendszer minden ciklusban önmagát leíró idősoros naplót állít elő.
Ezek együttesen biztosítják az elvárt funkcio nális helyességet és transzparens viselkedést.
