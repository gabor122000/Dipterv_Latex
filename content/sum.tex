\chapter{Összefoglalás és kitekintés}

A dolgozat célja egy nyílt forrású, konténerizált energetikai felügyeletre képes keretrendszer tervezése volt, 
ami alacsony költségű végponti eszközöket (ESP8266 mérő), Python-on alapuló vezérlő komponenst, 
idősoros adattárolást (Prometheus) és ezenkívül vizualizációt (Grafana) tartalmaz. A rendszer az interfészeken szétválasztja a mérést és a döntéshozatalt, 
a beavatkozást pedig ipari protokollon (Modbus/TCP) keresztül valósítja meg a villamos hálózati eszközökön. Az üzemeltetési környezet teszt állapotban, ahogy én is
használtam általában Docker Compose, 
nagyobb rendelkezésre állási igény esetén Kubernetes.

\paragraph{Eredmények és tanulságok.}
A szimulációs vizsgálatok azt mutatták, hogy a keretrendszer képes:
\begin{itemize}
  \item a mért villamos mennyiségek folyamatos, Prometheus-kompatibilis exportjára és azok megjelenítésére
  \item determinisztikus beavatkozásra, vezetéknélküli hálózaton keresztül,
        ez gyors reagálást és reprodukálhatóságot biztosít
  \item globális áramkeret pontos követésére, adott esetben a túllépések gyors csillapítására és az erőforrások igazságos újraelosztására a 
  max-min fair elv alapján
  \item skálázható, konténer-alapú üzemeltetésre, ami nagyban megkönnyíti a telepítést, a frissítést és a diagnosztikát.
\end{itemize}

Gyakorlati tapasztalat, hogy az egyszerű vezérlő rendszer fontosabb a túlbonyolított modellnél, 
a mérési zaj, a késleltetések és a végpontok nemlineáris viselkedése mellett a robusztus, determinisztikus vezérlő stabilabban teljesített. 
A komponensek laza csatolása és a metrika-alapú rendszer érdemben csökkenti a hibaelhárítás bonyolultságát.

\paragraph{A munka fő műszaki hozzájárulásai:}
\begin{enumerate}
  \item Egységesített mérési/vezérlési interfész energetikai rendszerekhez Prometheus-formátumú metrikákkal és REST és Modbus átalakítóval.
  \item Max-min fair elosztású beavatkozási szabályozó integrációja ipari rendszerbe.
  \item Konténer-alapú referenciaimplementáció Docker Compose vagy Kubernetes.
  \item Grafana-alapú üzemeltetési, illetve diagnosztikai irányítópultok, előre tervezett riasztások és alap-telemetria.
  \item Reprodukálható szimulációs rendszer, ami determinisztikusan teszteli a rendszer beállításait.
\end{enumerate}

\paragraph{Korlátok és érvényességi fenyegetések.}
A vizsgálatok kontrollált környezetben zajlottak; a terepi viszonyok (hálózati zavarok, különböző végponti firmware-ek, szélsőséges terhelési profilok) további kihívásokat jelenthetnek. A biztonsági réteg alapértelmezetten a helyi hálózatra és egyszerű hitelesítésre támaszkodik; nagyvállalati környezetben szükséges a végpont- és szolgáltatásoldali tanúsítványkezelés és kulcsforgatás beépítése. A Modbus/TCP protokoll korlátai (nincs beépített titkosítás, korlátozott hibakezelés) szintén megfontolandók.

\paragraph{Jövőbeli munka.}
A rendszer fejlesztésének lehetséges irányai:
\begin{itemize}
  \item \textbf{Biztonság és megbízhatóság:} mTLS alapú végpont-hitelesítés, kulcsforgatás, jogosultságkezelés.
  \item \textbf{Protokoll-tágítás:} IEC~60870-5-104 / IEC~61850 gateway, illetve modern ipari mezőbuszok támogatása a heterogén eszközparkhoz.
  \item \textbf{Fejlettebb szabályozás:} modellprediktív vagy hibrid (MPC + max-min) vezérlő vizsgálata időkorlátos optimalizálásra, 
  költség- és hálózati korlátok együttes kezelésére.
  \item \textbf{Edge-képességek:} lokális döntésképesség és \emph{graceful degradation} hálózati problémák esetén, OTA frissítési csatorna az ESP eszközökre.
  \item \textbf{Megfigyelhetőség és üzemeltetés:} trace-alapú hibakeresés (OpenTelemetry), automatikus eszköz-felfedezés, 
  konfiguráció- és verziókezelés.
  \item \textbf{Valós terepi pilot:} több, egymást zavaró terhelés és megosztott hálózati infrastruktúra mellett 
  végzett hosszú távú mérések, SLA-k és üzemeltetési költségek becslésével.
\end{itemize}

Az elkészült keretrendszer egy átlátható, bővíthető és költséghatékony alapot ad az energetikai felügyeleti rendszer elkészítéséhez. 
Az egyszerű és robusztus szabályozási elvek, a metrika-központúság és a konténeres üzemeltetés együtt olyan 
eszköztárat alkotnak, ami ipari környezetekben is gyors bevezetést 
és megbízható működést tesz lehetővé.
