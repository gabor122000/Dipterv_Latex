\chapter{Bevezetés}

A növekvő igények az elektromos energiafelhasználás hatékonyságára, 
a hálózatüzemeltetés biztonságára és az üzemeltetési költségek csökkentésére egyre 
inkább megkövetelik a fogyasztóhoz közeli hálózatrészek folyamatos megfigyelését és szükség 
esetén távoli irányítását. A korszerű épületüzemeltetésben 
ezért olyan megoldásokra van szükség, amelyek lehetővé teszik a \emph{mérést}, 
\emph{adataggregációt} és \emph{vezérlést}, továbbá az általam tervezett 
megoldás nyílt szabványokra és könnyen 
skálázható szoftveres komponensekre épül így csökkentve a költségeket. 
Dolgozatom célja egy ilyen, 
moduláris és konténerizált keretrendszer elkészítése energetikai 
felügyelethez, amely alacsony költségű végponti 
eszközöket (ESP8266 alapú mérő/vezérlő végpontokat), egy Python-alapú 
vezérlőrendszert, idősoros adatbázist (Prometheus) és vizualizációs 
felületet (Grafana) kapcsol össze.

A rendszer lényegi eleme egy \emph{water-filling}
allokációs elv alkalmazása az erőforrások igazságos szétosztására, amikor 
egy globális áramkeretet kell betartani. Ez a szabályozó 
előnyben részesíti a kis igényeket és a fennmaradó kapacitást egyenlően 
osztja el a nagy fogyasztók között. A megközelítés egyszerű, 
determinisztikus és robusztus, ezért választottam valós idejű erőforrás elosztásra.

A dolgozatban ismertetett megoldás \emph{nyílt forrású} és 
\emph{költséghatékony} alternatívát nyújt az ipari ökoszisztémákhoz 
képest. 
Bár ezek a gyártói rendszerek széles körű megfelelőséget és 24/7 támogatást kínálnak, 
a jelen keretrendszer előnye az \emph{átláthatóság}, a 
\emph{rugalmas bővíthetőség} és a \emph{gyors testreszabhatóság}, 
ami különösen kutatás-fejlesztési, oktatási és \emph{fejlesztési} 
célokra értékes.

A dolgozat céljai:
\begin{itemize}
  \item Egy egységesített, konténerizált keretrendszer tervezése és implementálása 
  energetikai felügyelethez.
  \item Végponti eszközök (ESP8266) 
  hatékony adatgyüjtése.
  \item Központi vezérlő megvalósítása Python/Flask alapokon, 
  REST API-val és Modbus/TCP beavatkozással.
  \item Igazságos terheléselosztás (water-filling) beépítése.
  \item A rendszer vizualizációja Grafanával.
\end{itemize}

Röviden a rendszer célja, 
hogy \emph{gyakorlatias, bővíthető és reprodukálható} mintát adjon energetikai 
felügyeleti rendszerekhez, megkönnyítse a mérés-döntés-beavatkozás 
zárt rendszerű megvalósítását nyílt szabványok és elterjedt eszközökkel.
