%----------------------------------------------------------------------------
\chapter{\bevezetes}
%----------------------------------------------------------------------------

A diplomatervem célja egy olyan keretrendszer megvalósítása, 
amely lehetővé teszi a távoli felügyeletet és vezérlést. 
A rendszer egyszerű felépítésű végponti elemeket 
(például ESP8266 alapú érzékelő- és vezérlőmodulokat) köt össze 
konténerizált vezérlőkomponenssel. 
A célja az adatok gyűjtése, tárolása és automatizált feldolgozása, 
valamint a végpontok megbízható kezelésének és redundanciájának 
biztosítása Kubernetes környezetben.
\paragraph{}
A villamosenergetikában az elmúlt években nagy igény jelentkezett 
a hálózati elemek, például autótöltők vagy megszakítók valós 
idejű felügyeletére és távoli vezérlésére. Ez fontos energiahatékonyság,
hálózati biztonság és a zavartűrés szempontjából is. 
A modern ipari rendszerekben alkalmazott szoftverek, 
mint a Schneider EcoStruxure Power Monitoring Expert vagy a Siemens SIMATIC 
Energy Suite széleskörű funkcionalitást biztosítanak, szabványoknak megfelelőséget 
és 24/7 gyártói támogatást kínálnak, 
de jelentősen drágábbak.
\paragraph{}
A munkámban bemutatni kívánt saját megoldás ezzel szemben nyílt 
forráskódú komponensekre épít (ESP8266 mikrokontrollerek, Prometheus 
idősoros adatbázis, Grafana vizualizáció és Python Flask vezérlőszerver), 
ez költséghatékony és jól testre szabható az előzőkkel szemben. 
A rendszerem moduláris felépítése miatt az érzékelők plug-and-play módon
csatlakoztathatóak és Kubernetes 
segíti a skálázhatóság, a redundancia és a magas rendelkezésre állást.

A dolgozat a következő részekből áll:

\begin{itemize}
    \item \textbf{Első fejezet:} A meglévő ipari megoldások ismertetése és összehasonlítása a saját fejlesztéssel.
    \item \textbf{Második fejezet:} A keretrendszer tervezésének bemutatása.
    \item \textbf{harmadik fejezet:} A rendszer egyes komponenseinek részletes megvalósítása.
    \item \textbf{Negyedik fejezet:} Nagy rendelkezésre állás megvalósítása hibrid klaszter topológiával.
\end{itemize}