\chapter{Szöveges interfészek a szimulációhoz}

A rendszer működésének igazolásához szükséges egy tesztkörnyezet, 
ami képes a valós hardver eszközök viselkedését szoftveresen emulálni. 
A fizikai tesztelés önmagában nem fedné le a szélsőséges terhelési eseteket 
(például a túlterhelését), és nem tenné lehetővé a hosszú távú, reprodukálható vizsgálatokat.

\section{Cél és áttekintés}
A szimulációs alrendszer vezérlése és adatgyűjtése
kizárólag szöveges interfészeken , gyakorlatban .txt fájlokon alapul.

Ezek biztosítják a kapcsolatot
a felhasználó és a futó környezet között:
a bemeneti fájlok definiálják a teszteseteket,
míg a kimeneti fájlok rögzítik az adatokat.

A cél, hogy a szimuláció kiegészítő eszközök
(pl. CSV-konverzió) nélkül,
egyszerű szövegszerkesztővel kiértékelhető legyen.

\noindent Rövid összefoglaló:
\begin{itemize}
\item \textbf{Bemenetek:} 
\begin{itemize}
  \item \texttt{thresholds.txt} - itt találhatóak meg a maximum és minimum értékek, 
  amit elérhet különböző pontjain a rendszer.
  \item \texttt{esp\{1..3\}\_schedule.txt} - ebben találhatóak meg a mérőpontok napirendjei.
  \item \texttt{sim\_control.txt} - itt találhatóak meg a futtatási állapotok parancsai.
\end{itemize}
  \item \textbf{Kimenet/napló:} \texttt{output.txt} (idősoros; egy sor = egy vezérlési ciklus)
  Ebben található meg az összes lényeges mérőszám és állapot minden ciklusban.
  \item \textbf{Webes felület:} ``Dev Panel'' (localhost:8080) a fájlok szerkesztéséhez, generálásához, 
  letöltéséhez, a futás indításához/megállításához, az idő nullázásához és a napló törléséhez használható.
\end{itemize}

\begin{figure}[H]
    \centering
    \includegraphics[width=1\textwidth]{figures/kibemenet.png}
    \caption{Interfészek}
    \label{fig:Interfészek}
\end{figure}

\section{Bemeneti szövegfájlok}

A vezérlő paramétereket és a szimulált környezetet
három szöveges állomány határozza meg.
Ezeket a fájlokat a rendszer
ciklikusan olvassa, így a módosítások újraindítás nélkül érvényesülnek.

\subsection{\texttt{thresholds.txt} -- küszöbök és maximum megengedhető áram}

Ez a fájl tartalmazza a globális határértékeket,
amiket a vezérlő algoritmusnak be kell tartania.
A fájl kulcs--érték párokat tartalmaz:

\begin{lstlisting}
# Kuszobertekek a vezerlo szerverhez
BREAKER_MAX_TOTAL=65.0    # [A] - Megszakito lekapcsolasi aram
BREAKER_MIN_TOTAL=35.0    # [A] - Megszakito bekapcsolasi aram
ALLOC_MAX_TOTAL=95.0      # [A] - Max aram ertek
\end{lstlisting}

\noindent\textbf{Feldolgozás módja:}
A kontroll Szerver minden vezérlési ciklus elején
beolvassa ezt a fájlt.
Ez lehetővé teszi, hogy szimuláció közben is változtathassuk
a rendelkezésre álló áramkeretet (\texttt{ALLOC\_MAX\_TOTAL}),
így tesztelve a szabályozó algoritmust.

\noindent Megjegyzések:
\begin{itemize}
    \item A \emph{megszakítók} (breakerek) logikája az \emph{aktuálisan mért hatásos} összáramhoz
    viszonyít (\texttt{BREAKER\_MAX\_TOTAL}, \texttt{BREAKER\_MIN\_TOTAL}).
    
    \item A SIM-ekre küldött korlátok (cap) a \emph{nyers igényekből} számítódnak \emph{max-min fair} elv
    szerint, az \texttt{ALLOC\_MAX\_TOTAL} keret figyelembevételével.
\end{itemize}

\subsection{\texttt{esp\{x\}\_schedule.txt} -- idősoros bemenet}

Ezek a fájlok írják le a szimulált fogyasztók (pl. autótöltők) viselkedését,
vagyis azt, hogy az idő függvényében mekkora áramigénnyel lépnek fel.
Az adott másodpercben + kívánt áram (A).
A menetrendben a legutóbbi időponthoz tartozó érték érvényes a következő megadásig.

\begin{lstlisting}
# seconds  amps
0         1.0
30        2.5
120       0.8
\end{lstlisting}

\noindent\textbf{Feldolgozás:}
Ezeket a fájlokat,
az \textbf{ESP szimulátorok} olvassák be.
A szimulátor a saját belső virtuális órájához igazodva
hozza létre a terheléseket.

\subsection{\texttt{sim\_control.txt} -- futtatási állapot}

Ez a fájl a szimuláció globális főkapcsolója.
Egyetlen szót tartalmazhat: \texttt{RUNNING} vagy \texttt{STOPPED}
(Az alapértelmezés \texttt{STOPPED}).

\noindent\textbf{Feldolgozás:}
Ez a fájl  egy szinkronizációs pont az egész rendszer számára.
\begin{itemize}
    \item A \textbf{Vezérlő} csak \texttt{RUNNING} állapotban futtatja az allokációs algoritmust
    és küld beavatkozó parancsokat.
    \item A \textbf{Szimulátorok} csak \texttt{RUNNING} állapotban léptetik a virtuális idejüket.
    Ha az állapot \texttt{STOPPED}-ra vált, a rendszer "lefagyasztja" a pillanatnyi állapotot,
    lehetővé téve a naplók és az állapotok statikus elemzését.
\end{itemize}

\section{Kimeneti szövegfájl}

\subsection{\texttt{output.txt} -- idősoros kimenet}
A vezérlő minden ciklusban \emph{egy sort} ír. A fájl alapértelmezetten \emph{append-only} 
a véletlen szerkesztést elkerülendő; a Dev Panel ``Clear output.txt'' művelete törli amennyiben ez szükséges, 
és a vezérlő legközelebb automatikusan újra létrehozza a fejlécet.

\noindent Formátum: \texttt{kulcs=érték} párok szóközzel elválasztva.
\begin{lstlisting}
# One record per line; fields are key=value separated by spaces
timestamp=1758199200 sim_state=RUNNING sum_current_amps=5.7 \
alloc_max_total_amps=6.0 max_total_amps=6.0 min_total_amps=1.0 \
sims=esp1:raw=2.0,effective=2.0,cap=2.0|esp2:raw=1.7,effective=1.7,
cap=2.0|esp3:raw=2.5,effective=2.0,cap=2.0 \
breakers=brk1:on,brk2:on
\end{lstlisting}

\noindent Kulcsok a kimeneti file-ban:
\begin{itemize}
  \item \texttt{timestamp} -- UNIX időpecsét (s).
  \item \texttt{sim\_state} -- globális állapot: \texttt{RUNNING}/\texttt{STOPPED}.
  \item \texttt{sum\_current\_amps} -- mért hatásos összáram (cap után).
  \item \texttt{alloc\_max\_total\_amps} -- allokációs keret (A).
  \item \texttt{max\_total\_amps} / \texttt{min\_total\_amps} -- breaker küszöbök (legacy nevek).
  \item \texttt{sims} -- \texttt{|} jellel szeparált lista mérési pontonként (SIM-enként):\\
  \texttt{espX:raw=\dots, effective=\dots, cap=\dots}\\
  ahol \texttt{raw} = menetrendi igény, \texttt{effective} = tényleges áram, \texttt{cap} = küldött maximum.
  \item \texttt{breakers} -- megszakítók állapota \texttt{on}/\texttt{off}, vesszővel elválasztva.
\end{itemize}

\section{Időkezelés és futtatás}
\begin{itemize}
    \item \textbf{RUNNING (Aktív futás):}
    Ilyenkor a virtuális óra minden ciklusban előrelép.
    A szimulált végpontok (ESP-k) a $t_{sim}$ időpillanat alapján keresik ki
    a menetrendjükből (\texttt{schedule.txt}) az aktuális áramigényt.
    A vezérlő szerver dolgozza fel az adatokat és
    számolja az allokációt, küldi ki az új korlátokat.
    
    \item \textbf{STOPPED (Felfüggesztés):}
    Ekkor a virtuális idő „befagy” ($t_{sim} = \text{const}$).
    A szimulátorok megálnak, tartják az utolsó beállított értéket.
    A vezérlő továbbra is mér és naplózza,
    de nem történik beavatkozás nem küld új korlátokat és nem kapcsol megszakítót.
\end{itemize}

\subsection{A STOPPED állapot szerepe a vezérlésben}

A rendszerben bevezetett \texttt{STOPPED} állapot a futás szüneteltetését szolgálja és
a szimulációs környezet konzisztenciáját segíti.
Bevezetésének kettő oka volt:

\begin{itemize}
    \item \textbf{Állapot-befagyasztás:}
    Mivel a vezérlő felfüggeszti az aktív beavatkozást,
    a rendszer pillanatnyi belső állapotát lehet ellenőrizni statikus körülmények között.
    
    \item \textbf{Determinisztikus újraindítás:}
    Ez az állapot a kiindulópontja a \texttt{RESET} műveletnek is,
    biztosítva, hogy minden komponens pontosan ugyanabból a $t=0$ időpillanatból induljon.
\end{itemize}

Ez a mechanizmus garantálja, hogy a tesztelés során
a vezérlő szoftver viselkedése determinisztikus és bármikor reprodukálható legyen.

\section{Reprodukálhatóság és feldolgozhatóság}
A bemenetek (küszöbök, menetrendek, futtatási állapot) verziózhatók és mellékelhetők. 
A kimeneti \texttt{output.txt} önleíró; minden rekord tartalmazza az adott ciklus lényeges paramétereit. 
A formátum egyszerűen feldolgozható bármely nyelven (kulcs=érték párok; \texttt{sims} és \texttt{breakers} 
mezők jól definiált szeparátorokkal).

\section{Rövid példa -- beállítás $\rightarrow$ kimenet}

\noindent\textbf{thresholds.txt}
\begin{lstlisting}
BREAKER_MAX_TOTAL=9.0
BREAKER_MIN_TOTAL=2.0
ALLOC_MAX_TOTAL=9.0
\end{lstlisting}

\noindent\textbf{esp1\_schedule.txt}
\begin{lstlisting}
# seconds  amps
0  50
60 10
\end{lstlisting}

\noindent\textbf{esp2\_schedule.txt}
\begin{lstlisting}
0  50
60 10
\end{lstlisting}

\noindent\textbf{esp3\_schedule.txt}
\begin{lstlisting}
0  50
60 100
\end{lstlisting}

\noindent Várható kiosztás a 0--60 s szakaszban: mindhárom SIM korlátozott, 
mivel az igény 150 A $>$ 9 A. 60 s után az 
igények \texttt{[10, 10, 100]} $\Rightarrow$ kiosztás \texttt{[10, 10, 70]}. A \texttt{cap} és 
az \texttt{effective} értékek ennek megfelelően jelennek meg az \texttt{output.txt}-ben.


% =====================================================================

\chapter{Fejlesztői panel (Dev Panel)}

\section{Cél és szerep}
A Dev Panel egy könnyű használatú webes felület, amely a szöveges bemenetek és kimenetek kezelését, 
a futtatás indítását/megállítását, az idő nullázását és a napló törlését teszi lehetővé. Célja 
a \emph{gyors kísérletezés} és a \emph{reprodukálható} tesztfutások támogatása külön eszközök nélkül.

\section{Architektúra áttekintése}
A panel egy Flask-alapú backendből (\texttt{/api/*}) és statikus frontendből (HTML+CSS+JS) áll. 
A backend közvetlenül a \texttt{./data} mappában található fájlokat kezeli, 
és hálózaton hívja az esp-t szimuláló konténerek végpontjait. A vezérlő külön, 
a saját portján (\(8000\)) fut; a Prometheus és Grafana eléréséhez gyorslinkek állnak rendelkezésre.

\section{Felhasználói felület és funkciók}

A grafikus felület két fő funkcionális egységre tagolódik:
a globális vezérlőkre (Simulation Control) és a tesztesetekre (Scenarios).

\subsection{Simulation Control (Vezérlőpult)}

Itt lehet a szimuláció globális állapotát menedzselni.

\begin{itemize}
    \item \textbf{Start/Stop:}
    A gombok a \texttt{sim\_control.txt} fájl írásával vezérlik a futtatást.
    \texttt{STOP} módban a szimulátorok virtuális ideje megáll,
    a vezérlő nem küld új parancsokat.
    
    \item \textbf{Reset (t=0):}
    A funkció leállítja a futást,
    és üzenettel nullázza minden szimulátor belső óráját,
    így a teszt biztosan nulláról indul.
    
    \item \textbf{Clear output:}
    Törli a naplófájl tartalmát,
    hogy a következő futás adatait ne keveredjenek a korábbiakkal.
    
    \item \textbf{Élő monitorozás:}
    A felület valós időben megjeleníti az aktuális összáramot és a rendszer státuszát.
\end{itemize}

\subsection{Scenarios (Szcenárió-kezelő modul)}

Lehetővé teszi a komplex tesztesetek egységes kezelését.
A modul három fülre bomlik:

\subsubsection{1. Presets (Előre definiált tesztek)}

A gyakran használt tesztesetek
(pl. statikus túlterhelés, hiszterézis-vizsgálat, dinamikus átrendeződés)
beépített sablonként érhetők el.

A felhasználó egyetlen kattintással betöltheti és elindíthatja ezeket.
Ez a funkció ("Auto Test Runner") gyorsítja a demonstrációt.

\subsubsection{2. Builder (Szerkesztő és Generátor)}

Ez a felület szolgál az egyedi tesztek összeállítására:

\begin{itemize}
    \item \textbf{Menetrend-generátor:}
    Segítségével grafikusan állíthatóak be a terhelési görbék
    (konstans, rámpa, lépcső, szinusz, random walk).
    Ezek azonnal megjelennek a szerkesztőablakban.
    
    \item \textbf{Thresholds beállítás:}
    A globális korlátok (\texttt{ALLOC\_MAX}, \texttt{BREAKER\_MAX})
    közvetlen szerkesztése.
    
    \item \textbf{Azonnali futtatás:}
    A "Run now" gombbal a beállított paraméterek mentésre kerülnek a fájlokba,
    és a szimuláció azonnal elindul az új értékekkel.
\end{itemize}

\subsubsection{3. Files (Fájlkezelés)}

A panel lehetőséget biztosít a teljes konfiguráció
(menetrendek + határértékek + vezérlési opciók)
JSON formátumú mentésére és visszatöltésére.

A "Saved Scenarios" listából a korábban elmentett tesztek
bármikor újra előhívhatók,
a "Downloads" szekció pedig lehetővé teszi a nyers bemeneti
és kimeneti (\texttt{output.txt}) fájlok letöltését archiválás céljából.

\section{Backend API interfész}

A Dev Panel backendje az alábbi REST végpontokon keresztül érhető el,
amelyeket a frontend vagy külső automatizációs szkriptek is használhatnak:

\begin{table}[h!]
\centering
\begin{tabular}{|l|l|}
\hline
\textbf{Végpont} & \textbf{Funkció} \\ \hline
\texttt{GET /api/sim\_state} & A globális állapot (RUNNING/STOPPED) lekérdezése \\ \hline
\texttt{POST /api/sim\_state} & A futtatási állapot módosítása \\ \hline
\texttt{POST /api/reset\_sim\_time} & Minden szimulátor órájának nullázása \\ \hline
\texttt{POST /api/clear\_output} & A naplófájl törlése \\ \hline
\texttt{POST /api/run\_scenario} & Teljes teszteset (JSON) alkalmazása és indítása \\ \hline
\texttt{POST /api/save\_scenario\_json} & Aktuális beállítások mentése a szerverre \\ \hline
\texttt{GET /api/read /api/write} & Nyers fájlműveletek a \texttt{/data} mappában \\ \hline
\end{tabular}
\caption{A Dev Panel legfontosabb API végpontjai}
\label{tab:devpanel-api}
\end{table}

\begin{figure}[H]
    \centering
    \includegraphics[width=1\textwidth]{figures/devpanel.png}
    \caption{Devpanel}
    \label{fig:devpanel}
\end{figure}