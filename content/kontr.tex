\chapter{Központi vezérlő felépítése}

\section{Flask alapú vezérlő}

A központi vezérlő a rendszer "agya", amely a Kubernetes cluster-en belül, 
konténerizált környezetben fut. A szoftver fejlesztéséhez a \textbf{Python} programozási nyelvet és 
a \textbf{Flask} keretrendszert választottam, mivel ezek rugalmas és gyors fejlesztést tesznek lehetővé, 
valamint kiváló támogatást nyújtanak a RESTful API-k megvalósításához.

A vezérlő architektúrája három fő rétegre bontható:
\begin{enumerate}
    \item \textbf{Kommunikációs réteg:} A bejövő HTTP kéréseket (mérésadatok) és a 
    kimenő válaszokat (aktuális áramlimitek) kezeli. 
    Az Ingress Gateway felelős a forgalom megfelelő pod-hoz való irányításáért.
    \item \textbf{Adatkezelési réteg:} A beérkező mérési adatokat a rendszer a Prometheus idősoros 
    adatbázisba továbbítja, illetve onnan olvassa ki a historikus adatokat a döntéshozatalhoz.
    \item \textbf{Logika réteg:} Itt történik a tényleges számítás és döntéshozatal.
\end{enumerate}

Ez az  alfejezet kizárólag a szoftverkomponensek technikai felépítését és az adatkapcsolatokat mutatja be. 
A vezérlő tényleges döntési mechanizmusát, az alkalmazott \textit{water-filling} algoritmust és 
az allokációs stratégiákat részletesen a következő alfejezetben fejtem ki. 
A rendszer valós idejű működésének korlátait, különös tekintettel a hálózati késleltetésre (latency) 
és a szabályozási köridőre, az ezt követő alfejezet tárgyalja.

\subsection{A Flask alkalmazás felépítése}
A Python alkalmazás alapja egy REST API interfész, 
ami Get kéréseket küld a lokális vezérlőknek. 
Minden kérés tartalmazza az adott végpont azonosítóját 
és az általa mért pillanatnyi áramfelvételt. 
Az alkalmazás állapotmentes módon működik, az állapotokat a Prometheus adatbázis és a 
memóriában tárolt rövid távú cache kezeli.

\section{Max--min fair elosztás}
\label{chap:max-min-fair}
\subsection{Elméleti háttér és cél}

\subsubsection{A szabályozási feladat és definíciók}

A dolgozat rendszerének fő feladata, hogy a rendelkezésre álló, 
korlátos villamos teljesítményt dinamikusan ossza el a fogyasztók (esetemben elektromos autótöltők) között úgy, 
hogy a hálózat fizikai védelme ne oldjon le, így a felhasználható áramon belül maradva maximalizáljuk 
a lehetőségeket.

Az elosztási algoritmus ismertetéséhez, szükséges definiálni a 
rendszerparamétereket, amik a vezérlő működésének alapját képzik. 
A rendszerben két védelem található:

\begin{itemize}
    \item Fizikai korlát (\texttt{BREAKER\_MAX\_TOTAL})
    \item Allokációs keret (\texttt{ALLOC\_MAX\_TOTAL})
\end{itemize}

A bemutatott \emph{max--min fair} algoritmus fő feladata tehát az, 
hogy a fogyasztók igényeit úgy elégítse ki, hogy összegük soha ne lépje át a 
szoftveres \texttt{ALLOC\_MAX\_TOTAL} keretet.

\subsubsection{Motiváció és cél}

A szimulált fogyasztók  áramigénye (\(d_i\)) időben változik. 
Adott egy globális, maximum áramérték \(\,B=\texttt{ALLOC\_MAX\_TOTAL}\,\) amperben, 
ennél a tényleges összáram nem lehet nagyobb. A cél egy olyan kiosztás \(\,a_i\) meghatározása, amely
(i) nem lépi túl az egyes igényeket (\(0\le a_i\le d_i\)),
(ii) a teljes kereten belül marad (\(\sum_i a_i \le B\)),
(iii) és \emph{fair} a kis igényűekkel szemben, azaz a kis igények teljesülnek először, a 
fennmaradó kapacitás pedig egyenlő alapról oszlik meg.

\subsubsection{Definíció}
Egy \(\,a=(a_1,\dots,a_n)\) kiosztás \emph{max--min fair}, ha bármely más megengedett \(\,y\) esetén, 
ha létezik \(i\) úgy, hogy \(y_i > a_i\), akkor létezik \(j\) olyan, hogy \(a_j \le a_i\) és \(y_j < a_j\). 
Intuíció: csak a \emph{már kisebb} részesedésűek rovására lehet növelni bárki juttatását. \cite{leboudec_tutorial}

\subsubsection{Feltöltés}
A max--min fair kiosztás felírható egyetlen paraméterrel:
\begin{equation}
  a_i = \min\{\,d_i,\ \lambda\,\}, \qquad \text{ahol }\ \sum_{i=1}^n \min\{d_i,\lambda\} \;=\; B.
\end{equation}
A \(\lambda\) \emph{vízszint} úgy választandó, hogy a keret pont kiteljen (vagy ha \( \sum_i d_i < B\), 
akkor \(\lambda\ge \max_i d_i\), vagyis nincs korlát).

\subsubsection{Algoritmus és bonyolultság}
Gyakorlati, determinisztikus eljárás (progresszív töltés):
\begin{enumerate}
  \item Rendezzük az igényeket növekvő sorrendbe: \(d_{(1)} \le \dots \le d_{(n)}\).
  \item Iteráljuk \(k=1..n\): feltételezzük, hogy az első \(k\) igény teljesül (\(a_{(i)}=d_{(i)}\), \(i\le k\)), 
  és a maradék \(B_k = B - \sum_{i=1}^k d_{(i)}\) egyenlő szinten oszlik meg a még nyitott \(n-k\) elemre. 
  A jelölt vízszint: \(\lambda_k = B_k/(n-k)\).
  \item Ha \(\lambda_k \le d_{(k+1)}\), megtaláltuk a vízszintet: az összes hátralévő \(a_{(i)}=\lambda_k\) 
  (és a korábbiak \(d_{(i)}\)).
  \item Ha minden \(d_{(i)}\) teljesül és még marad keret, akkor nincs korlátozás: \(a_i=d_i\).
\end{enumerate}
A rendezés miatt az időbonyolultság \(O(n\log n)\). A megvalósított vezérlőben egy ekvivalens, 
iteratív \emph{progresszív} algoritmus fut, amely kis elemszámon szintén gyors és stabil.

\begin{figure}
    \centering
    \includegraphics[width=1\textwidth]{figures/Principle-of-Water-Filling-algorithm-76.png}
    \caption{Water-filling elve telekomunikációban. \cite{Slacik2021Sensors}}
    \label{fig:water-filling}
\end{figure}

\subsubsection{Tulajdonságok}
\begin{itemize}
  \item \textbf{Egyenlő szint elve:} a \(\lambda\) alatti igények teljes, a \(\lambda\) 
  felettiek \(\lambda\)-ig kapnak. Így a kis igényűek sosem szenvednek hátrányt.
  \item \textbf{Monotonitás:} ha a keret \(B\) nő, akkor \(\lambda\) nem csökken, és senki kiosztása nem csökken.
  \item \textbf{Határhelyzetek:} ha \(\sum_i d_i \le B\) → nincs cap (végtelen korlát). Ha \(B=0\) → minden \(a_i=0\).
\end{itemize}

\subsection{A vezérlőben alkalmazott megvalósítás}

\subsubsection{Kapcsolat a rendszer komponenseivel}
A vezérlő igényekből (\texttt{raw\_current}) számolja a limiteket a fenti elv szerint 
a \texttt{ALLOC\_MAX\_TOTAL} kereten. A \emph{megszakító} (breaker) logika ettől független, 
a \emph{mért, tényleges} áramhoz viszonyít (\texttt{BREAKER\_MAX\_TOTAL}, \texttt{BREAKER\_MIN\_TOTAL}) 
biztonsági rétegként.

\subsubsection{Példák}
\paragraph{Klasszikus példa.}
\(d=[10,10,100]\), \(B=90\) \(\Rightarrow\) \(a=[10,10,70]\) (a két kicsi teljesül, a maradék egy szinten oszlik meg).
\paragraph{Vegyes igények.}
\(d=[3,8,8,20]\), \(B=25\) \(\Rightarrow\) rendezve az első igény (3) teljesül, a maradék \(22\) 
három felé oszlik: \(a=[3,\,7.33,\,7.33,\,7.33]\) A.

\subsubsection{Implementációs részletek}
A limitek csak \(\pm 10^{-3}\) A változás felett frissülnek a fogyasztók felé (zajcsillapítás), 
a „nincs korlát” állapotot nagy \(\texttt{INF\_CAP}\) érték reprezentálja. Ha a nyers igény összeg a keret alá esik, 
a limitek feloldódnak.

\subsubsection{Alternatív allokációs stratégiák}
A globális keret elosztására több lehetőség is fennáll. A megfelelő algoritmus 
választása sokat tud javítani a rendszer hatékonyságán, ezért nagyon fontos. A \texttt{max--min fair} 
elv mellett kettő alternatívát találtam.

\paragraph{Proporcionális elosztás.}
Ennél a megközelítésnél minden fogyasztó a teljes igény (\(\sum_j d_j\)) arányában részesül a 
rendelkezésre álló keretből, amennyiben \(\sum_j d_j > B | B = Globál keret\). A kiosztás képlete: 
\( a_i = d_i \times (B / \sum_j d_j) \).
\begin{itemize}
    \item \emph{Példa:} Ha a keret \(B=90\)\,A, és az igények \(d=[50, 50, 100]\), 
    a teljes igény itt ekkor, \(\sum d_j = 200\)\,A. Az arányos kiosztás 
    \(a \approx [22.5, 22.5, 45]\)\,A lenne.
    \item \emph{Előnye:} Az egyik legegyszerűbb algoritmus.
    \item \emph{Hátrány:} Ez a stratégia a "nagy" fogyasztóknak 
    kedvez, és a kis igényűeket "éhezteti" (starvation).
\end{itemize}


\paragraph{Prioritásos (súlyozott) elosztás.}
Ebben az esetben minden fogyasztó kap \(w_i\) prioritási szintet 
(súlyt). Ha túlterhelés lépne fel a rendszer először a magasabb prioritású 
fogyasztókat elégíti ki, és a maradék keretből kapnak az alacsonyabb 
prioritású fogyasztók.
\begin{itemize}
    \item \emph{Előny:} Lehetővé teszi a "prémium" ügyfél fogalmát pl.: adott emberek 
    autótöltője mindig kap áramot.
    \item \emph{Hátrány:} Szintén bevezeti az "éheztetés" (starvation) problémáját, ahol egy 
    alacsony prioritású végpont akár soha nem kap erőforrást.
    A rendszer komplexitása nő, és a konfiguráció nehézkessé válik.
\end{itemize}


\paragraph{A választás indoklása.}
Az alternatívákkal szemben azért esett a választás a \textbf{max--min fair} elosztásra mert ez
biztosítja a legjobb egyensúlyt az egyszerűség és az igazságosság között. 
Motivációk között megfogalmazott pontokat teljesíti:

\begin{enumerate}
    \item \emph{Védi a kis fogyasztókat:} Garantálja, hogy aki keveset kér, az 
    megkapja, amíg a keret ezt egyáltalán lehetővé teszi. Ezzel elkerüli az 
    arányos elosztás fő problémáját az "éheztetést".
    \item \emph{Igazságos a nagy fogyasztók között:} A fennmaradó kapacitást 
    egyenlően osztja szét a nagy igényű fogyasztók között, anélkül, hogy 
    bonyolult prioritási problémákat kéne megoldani.
    \item \emph{Determinisztikus és stabil:} Az algoritmus viselkedése kiszámítható, 
    és könnyen implementálható.
\end{enumerate}
A tárgyalt tulajdonságok miatt esett a választás a Max--Min fair elvre.

\section{Vezérlési késleltetés és rendszer-reakcióidő}

\subsection{A reakcióidő kritikus szerepe}

A dolgozatban bemutatott keretrendszer nem csak egy statikus 
eszköz, hanem egy aktív beleszól a rendszer elemek működésébe
Ezentúl vannak egyéb fizikai eszközök amik a felügyeleti rendszertől függetlenül üzemelnek. 

Energetikai rendszerek felügyeleténél a reakcióidő kritikus tényező, 
mert egy zárlat esetén rövid idő alatt nagyon nagy energiák szabadulnak fel. 
A keretrendszer célja, hogy a szoftveres energia 
allokációval a teljes keretet(\texttt{ALLOC\_MAX\_TOTAL}) úgy ossza fel,
hogy megelőzze a fizikai védelmi eszközök
(megszakítók)
leoldását. A rendszer versenyt fut az idővel: a szoftvernek gyorsabban 
kell beavatkoznia, mint ami a fizikai 
megszakító védelmén van beállítva leoldási időnek.
Ha a késleltetés túl nagy, a megszakítók leoldanak, mielőtt 
a szoftveres korlátozás érvénybe léphetne. 

\subsection{A vezérlés komponenseinek késleltetése }

A rendszer teljes reakcióidejének ($T_{total}$) elemzéséhez szükséges a 
vezérlési lánc komponensekre bontása, és az egyes elemek által okozott 
rész-késleltetések azonosítása. A beavatkozási folyamat a túlterhelés 
észlelésétől az elhárításig, az alábbi lépésekből épül fel:

\begin{enumerate}
    \item \textbf{Mérés (\(T_{measure}\)):} Az \texttt{ESP8266} 
    árammérő szenzora (pl. SCT-013) fizikailag megméri az áramot. 
    Ez kb. valós időben törtnénik, ez a 
    késleltetés elhanyagolható.

    \item \textbf{Adat küldés (\(T_{publish}\)):} Az \texttt{ESP8266} 
    frissíti a mért értéket és elérhetővé teszi azt 
    a \texttt{/metrics} végponton. Ez elég szintén elég gyors, hogy valós idejűnek nevezzük .

    \item \textbf{Adatgyűjtés (\(T_{scrape}\)):} A Prometheus szerver a központi vezérlőn keresztül,
    gyűjti az adatokat. Mivel a központi veérlővel szinkronban kérdeznek, ezért szükséges mindekettőnek 
    a konfigurációjában (\texttt{scrape\_interval}) változót előre  és egységesen meghatározni, 
    ez időközönként lekérdezi a mérő végpontjának a \texttt{/metrics} URL-jét. 
    \emph{Ez a rendszer egyik legjelentősebb késleltetési tényezője.} 
    Ha egy tipikus 5 másodperces időközről beszélünk a mérés és a 
    Prometheus általi észlelés között, átlagosan 2.5, 
    rosszabb esetben 5 másodperc is eltelhet.

    \item \textbf{Adatfeldolgozás (\(T_{query}\)):} A 
    Központi vezérlő lekérdezi az adatbázisból az 
    áramokat. 
    A lekérdezés gyors, viszont ez a lépés 
    is a Prometheus belső frissítési 
    ciklusától függ.

    \item \textbf{Döntéshozatal (\(T_{algo}\)):} A Python szerver 
    az adatokra lefuttatja a \texttt{max--min fair} 
    algoritmust. Az elosztást tagalaló fejezetben láttuk, hogy ennek bonyolultsága 
    \(O(n \log n)\), ez kis (\(n=3..10\)) fogyasztószám esetén 
    elhanyagolható.

    \item \textbf{Beavatkozás (\(T_{command}\)):} A szerver a számolás után 
    korlátokat küld a végpontoknak. Itt egyedül a hálózati késleltetések lépnek fel 
    (Wi-Fi, LAN).

    \item \textbf{Végrehajtás (\(T_{execute}\)):} Az \texttt{ESP8266} 
    fogadja a parancsot és a Modbus-os fizikai eszközök beállítják az autótöltő vagy egyéb eszköz maximális 
    áramát. Ez is egy gyors műveletnek számít.
\end{enumerate}

Így jön ki, hogy a következő lesz a teljes késleltetés: 
\begin{equation}
    T_{total} = T_{measure} + T_{publish} + T_{scrape} + T_{query} 
              + T_{algo} + T_{command} + T_{execute}
\end{equation}

\begin{figure}[!ht]
    \centering
    \includegraphics[width=0.8\textwidth, keepaspectratio]{figures/Code_Generated_Image.png}
    \caption{késleltetés eloszlása} 
\end{figure}

\subsection{A fő késleltetési tényezők}

Az előző pont lánca jól mutatja, hogy a késleltetést nem az algoritmus
limitálja, hanem az adatgyűjtés architektúrája.

A teljes késleltetés nagy részét  (\(T_{total}\)) kettő tényező dominálja:
\begin{enumerate}
    \item \textbf{Prometheus Scrape Interval (\(T_{scrape}\)):} 
    Ez a rendszer "szívverése". A vezérlőhurok nem tud gyorsabban 
    reagálni, mint amilyen sűrűn friss adathoz jut. Ha ez az 
    érték 5 másodperc, a rendszer fizikailag "vakon" repül 
    5 másodpercig, és csak utólag értesül a korábbi eseményekről.
    
    \item \textbf{Hálózati késleltetés (\(T_{command}\)):} 
    A Wi-Fi-alapú kommunikáció további, változó hosszúságú (jitter) 
    késleltetést vihet a rendszerbe, terhelt vagy 
    zajos hálózati környezetben, ez különösen problémás hiszen
    nehéz számolni a dinamikusan változó késleltetéssel.
\end{enumerate}

Míg a hálózati késleltetés általában tíz-száz milliszekundum 
nagyságrendű, a \texttt{scrape\_interval} alapbeállítása 
másodperces nagyságrendű. A tesztek során a hálózat kímélése érdekében ezt a nagyobb 
alapértéket használtam, azonban éles üzemben ez lecsökkenthető akár $100~\text{ms}$-ra is. 
Ez bár növeli a hálózati forgalmat és a szerver terhelését, jelentősen csökkenti a teljes késleltetést. 
Így a megszakítók védelmi szintje beállítható egy biztonságos, például $500~\text{ms}$-os értékre, anélkül, 
hogy aggódnunk kellene a késleltetésből fakadó téves fizikai leoldások miatt.

\subsection{Védelmi beállítások késleltetések alapján}

A késleltetés számítása után lehetséges beállítani a védelem értékeit. 
A rendszerben kettő kvázi független védelmi szint létezik:

\begin{itemize}
    \item \textbf{1. Szoftveres Védelem (Allokáció):} 
    Ezt a \texttt{ALLOC\_MAX\_TOTAL} kerettel tudjuk beállítani. Célja a 
    \emph{komfort} és a \emph{hatékonyság} biztosítása, a keretek 
    a felhasználható teljesítmény folyamatos kihasználása és 
    a fizikai leoldások \emph{megelőzése}. Reakcióideje megfelelően pár száz milliszekundum (\(T_{total}\)).
    
    \item \textbf{2. Fizikai Védelem (Megszakító):} 
    Ezt a megszakító védelmének fizikai beállítása határozza meg. 
    Célja a hálózat fizikai védelme. 
    Reakcióideje azonnali viszont a védelemben késleltetést állítunk be, amit körülbelül
    a szoftveres érték kétszeresére állítunk.
\end{itemize}

A rendszer stabil működésének feltétele, hogy e két szint 
megfelelően legyen összehangolva, a kétszeres érték valószínűleg egy megfelelő mértékű
eltávolítás, ami enged helyet még a jitter-nek és egyéb hibáknak a vezeték nélküli hálózaton. 

\paragraph{Konfigurációs példa.}
Tegyük fel, hogy:
\begin{itemize}
    \item A szoftveres keret \(\texttt{ALLOC\_MAX\_TOTAL} = 90\,A\).
    \item A Prometheus \texttt{scrape\_interval} és a teljes hurok 
          késleltetése \(\textbf{\(T_{total}\) = 200\,ms}\).
    \item A csatlakoztatott eszközök (pl. autótöltők) képesek 
          200 milliszekundum alatt további 20\,A terhelést bekapcsolni 
          (pl. egy új autó csatlakozik).
    \item Ebben az esetben a fizikai megszakító védelmi beállítása 
            legyen legalább \(\textbf{\(T_{total}\) = 400\,ms}\).
\end{itemize}

Ebben az esetben az esetben megfelelő a kettő késleltetés távolsága, hogy ne történjen véletlen leoldás.