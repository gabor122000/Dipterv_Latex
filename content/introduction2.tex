\chapter{Bevezetés}

% ----------------------------------------------------------------
% 1. GLOBÁLIS ENERGETIKAI KONTEXTUS
% ----------------------------------------------------------------

Az elmúlt évtizedben az energetikai szektor, és különösen a villamosenergia-rendszer, 
alapvető változásokon ment keresztül. 
A megújuló energiaforrások - elsősorban a nap- és szélenergia - 
exponenciális terjedése, valamint az elektromobilitás megjelenése 
új kihívások elé állította a hálózatüzemeltetőket. 
A hagyományos, egyirányú energiaáramlásra méretezett hálózatoknak 
ma már dinamikus, kétirányú terheléseket kell kezelniük, 
ahol a fogyasztás és a termelés egyensúlya folyamatosan ingadozik.

A jövő energetikája adatvezérelt: a mérés, az adatfeldolgozás és 
a beavatkozás ciklusidejének drasztikus csökkenése a kulcs 
a fenntartható üzemeltetéshez.

% ----------------------------------------------------------------
% 2. MOTIVÁCIÓ ÉS A PIACI KÖRNYEZET (IT/OT KONVERGENCIA)
% ----------------------------------------------------------------

\section{Motiváció és ipari környezet}

Az épületautomatizálás és az ipari folyamatirányítás területén 
jelenleg egy érdekes kettősség figyelhető meg. 
A piacon domináns nagyvállalati megoldások (például Siemens, Schneider Electric) 
robusztus, szabványosított hardvert és szoftvert kínálnak. 
Ezek a rendszerek zárt ökoszisztémákra épülnek. 
Magas licencköltségeik, egyedi kommunikációs protokolljaik és 
a gyártói függőség gyakran gátat szabnak a kisebb felhasználóknak.

Ezzel párhuzamosan, az Ipar 4.0 rendszerek részeként, 
az olcsó mikrokontrollerek és beágyazott rendszerek
ma már olyan számítási kapacitással és kommunikációs képességekkel rendelkeznek, 
amelyek néhány éve még csak drága ipari PC-k kiváltságai voltak. 
Ez a technológiai demokratizálódás teremt alapot dolgozatom motivációjának: 
lehetséges-e ipari szemléletű, megbízható felügyeleti rendszert építeni 
ezen költséghatékony, nyílt eszközök felhasználásával?

% ----------------------------------------------------------------
% 4. CÉLKITŰZÉSEK
% ----------------------------------------------------------------

\section{A dolgozat célkitűzései}

Jelen diplomaterv célja egy olyan moduláris, nyílt forráskódú keretrendszer 
tervezése és megvalósítása, amely alternatívát kínál 
a drága ipari megoldásokkal szemben kutatási és fejlesztési célokra. 
A rendszernek képesnek kell lennie a teljes mérési-beavatkozási lánc kezelésére.

A megvalósítás során az alábbi konkrét célokat tűztem ki:

\begin{itemize}
    \item \textbf{Architektúra tervezés:} 
    Egy egységesített, konténerizált szoftverkörnyezet kialakítása 
    a következő elemekkel 
    idősoros adatgyűjtés (Prometheus), vezérlés (Python) 
    és a vizualizáció (Grafana).
    
    \item \textbf{Végponti integráció:} 
    Költséghatékony IoT eszközök (ESP8266) illesztése a rendszerbe 
    szabványos ipari (Modbus/TCP) és webes (REST API) protokollokon keresztül.
    
    \item \textbf{Algoritmikus vezérlés:} 
    A water-filling allokációs algoritmus implementálása.
    
    \item \textbf{Validáció és vizualizáció:} 
    A rendszer működésének igazolása valós idejű mérésekkel, 
    valamint egy átlátható kezelőfelület létrehozása az üzemeltető számára.
\end{itemize}

A dolgozat bemutatja, hogy a nyílt szabványok és a modern szoftvertechnológiák 
megfelelő kombinációjával létrehozható egy olyan rugalmas energetikai menedzsment rendszer, 
amely funkcióban hasonló, mint a drágább ipari megoldásokat.