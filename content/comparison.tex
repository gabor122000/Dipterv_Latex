\chapter{Piaci megoldások áttekintése}

\section{Meglévő ipari megoldások}

A piaci megoldások bemutatására 
két piacvezető gyártó, a Schneider Electric és a Siemens
rendszereit választottam elemzésre.

Ezek a rendszerek bemutatása azért indokolt,
mert jelenleg ők képviselik az ipari sztenderdet
az energetikai felügyelet területén.
Ezek adják a referenciaalapot,
amelyhez viszonyítva reálisan értékelhető
a dolgozatban bemutatott saját, költséghatékony fejlesztés
teljesítménye és korlátai.

\subsection{Schneider Power Monitoring Expert}

A Schneider Electric kínálatából a Power Monitoring Expert szoftvert vizsgálom,
ami a kritikus energiaellátású létesítmények felügyeletére készült.
A rendszer célja, hogy átláthatóságot biztosítson
az energiaelosztó hálózatban.\cite{sePME}

\subsubsection{Alapfunkciók}

\begin{itemize}
    \item Segít csökkenteni a meddő teljesítmény termelést és az ebből keletkező büntetéseket.
    \item Saját számlát készít, a helyi mérések alapján, hogy összehasonlítási alap legyen a számlákhoz.
    \item Segít elszámolhatóságot biztosítani alszámlázáshoz.
    \item Berendezések teljesítményét és várható élettartamát ellenőrzi.
    \item Valós idejű adatfigyelés, riasztás és energiafolyamatok vezérlése a létesítményen belül.
    \item Azonosítsa a potenciális áramminőségi problémákat a hálózatában, és értesíti erről a személyzetet.
\end{itemize}


\begin{figure}[ht]
    \centering
    \includegraphics[width=0.3\textwidth]{figures/sePME.png}
    \caption{Schneider Electric PME model\cite{sePME}}
    \label{fig:schneider-pme}
\end{figure}

\subsubsection{Előnyök}

Az energiamérési rendszer használata átlagban 24\%-kal csökkentette a fogyasztást, 
és 30\%-al a költségeket.

Mivel folyamatos megfigyelés és  beavatkozás lehetséges, a problémák korai szakaszában orvosolhatóak így
ezeket 22\%-al lehet csökkenteni. Ez a tudatosság csökkenti a hiba utáni visszaállítások idejét is.
Ezenkívül segít a mögöttes problémák megtalálásában is.\cite{sePME}

\subsection{Siemens SIMATIC Energy Suite}

Hasonló helyzetben áll a piacon a Siemens megoldása,
a SIMATIC Energy Suite,
ez közvetlenül a gyártóautomatizálási környezetbe integrálódik.
A rendszer hasonló funkciókkal rendelkezik mint a Schneider rendszere.

\subsubsection{Alapfunkciók}

A Siemens SIMATIC Energy Management rendszere integrált tehát nem csak megfigyelésre alkalmas hanem vezérlésre is. 
A már létező TIA Portal keretrendszerükbe épül és így egy helyen elérhető a többi rendszerükkel. 
Ez szintén egy moduláris és skálázható rendszer.
Megfelel az ISO 50001 szabványnak, és ez is alkalmazható terhelés figyelésre számlázásra és rendszerelemzésre,
mint az előzőleg taglalt rendszer.\cite{sieEMS}

\subsubsection{Előnyök}

\begin{itemize}
    \item Terepi szintű integráció saját és más eszközökkel. Figyelve itt az egyedi eszközökre.
    \item Gyártás szintű felügyelet. Üzem szintű energia fogyasztást lehet vele figyelni.
    \item Nagyobb rendszerekben vállalati szintű energiaelemzés, ahol több helyszín között is lehet felügyelni.
    \item Ezentúl alkalmas beavatkozásra is. Amennyiben túl nagy a fogyasztás képes fogyasztókat leválasztani távolról is akár.
\end{itemize}

\begin{figure}
    \centering
    \includegraphics[width=0.8\textwidth]{figures/sieEMS.png}
    \caption{Siemens EMS model\cite{sieEMS}}
    \label{fig:siemens-ems}
\end{figure}



\section{Nyílt forráskódú és közösségi megoldások}

Az ipari rendszerek mellett az elmúlt években megjelentek 
a rugalmasabb, alacsonyabb költségű, jellemzően a "Smart Home" 
és a kisvállalati szegmensre fókuszáló nyílt megoldások is. 
Ezek nem rendelkeznek ipari tanúsítványokkal, 
viszont egyszerűségük miatt jó összehasonlítási alapot képeznek.

\subsection{Home Assistant}

A Home Assistant jelenleg az egyik legnépszerűbb nyílt forráskódú 
otthonautomatizálási platform. 
Fő erőssége, hogy lényegében bármilyen IoT eszközt képes integrálni, 
ami lehetővé teszi többek között energetikai mérők kezelését is.  \cite{HomeAssistant}

\begin{itemize}
    \item \textbf{Előnyök:} Ingyenes, nagy közösség, 
    helyi működés, vizuális felület.
    \item \textbf{Hátrányok:} Főként otthoni felhasználásra tervezték, 
    az idősoros adatok hosszú távú tárolása és elemzése 
    korlátozott, 
    nem specifikusan energetikai szabályozásra lett kitalálva.
\end{itemize}

\subsection{OpenEnergyMonitor}

Az OpenEnergyMonitor projekt kifejezetten az energetikai mérésekre specializálódott. 
Szoftveres központja, az Emoncms egy webes alkalmazás 
elektromos adatok feldolgozására és vizualizációjára. \cite{EmonCMS}

\begin{itemize}
    \item \textbf{Előnyök:} Ez már energia-fókuszú.
    \item \textbf{Hátrányok:} A vezérlési funkciók 
    nem túl hangsúlyosak.
    A hardveres rendszer kötöttebb.
\end{itemize}

\section{A vizsgálat tanulságai és a tervezési követelmények}

A piaci körkép alapján látható, hogy létezik egy szakadék 
a drága, zárt ipari rendszerek és az általános célú hobbi megoldások között. 
Míg az ipari eszközök garantálják a pontosságot és a szabványosságot, 
addig költségvonzatuk megnehezíti használatukat kisebb projektekben.

A saját rendszeremmel szemben ezért nem cél a Siemens vagy Schneider 
megoldásaival való közvetlen verseny. 
A cél egy olyan köztes megoldás létrehozása, 
ami ötvözi a nyílt rendszerek rugalmasságát 
egy ipari jellegű szabályozási logikával.

Ez alapján a saját fejlesztésű rendszerrel szemben 
az alábbi \textbf{elvárásokat} fogalmaztam meg:

\begin{itemize}
    \item Modularitás és Nyíltság
    \item Költséghatékonyság
    \item Aktív beavatkozási képesség
    \item Reprodukálhatóság
\end{itemize}

Összefoglalva: a tervezett rendszer a csináld magad (DIY) árkategóriában 
kíván megvalósítani egy, a funkcióit tekintve az ipari rendszereket megközelítő, 
zárt szabályozási rendszert. \ref{tab:rendszer_osszehasonlitas}. Táblázat

\newcolumntype{P}[1]{>{\raggedright\arraybackslash}p{#1}}

\begin{table}[ht]
    \centering
    \small % Kicsit kisebb betűméret, hogy minden kényelmesen elférjen
    \renewcommand{\arraystretch}{1.4} % Nagyobb sorköz a jobb olvashatóságért
    
    % X oszlopok automatikusan tördelik a szöveget a rendelkezésre álló helyen
    \begin{tabularx}{\textwidth}{|l|X|X|X|}
    \hline
    
    \textbf{Jellemző} & 
    \textbf{Tervezett rendszer} & 
    \textbf{Schneider \newline PME} & 
    \textbf{Siemens \newline Energy Suite} \\ 
    \hline
    \hline
    
    % --- TECHNOLÓGIAI ALAPOK ---
    \multicolumn{4}{|c|}{\textit{Technológiai alapok}} \\
    \hline
    
    \textbf{Eszközök} & 
    ESP8266 mikrovezérlő \newline + Áramváltó (CT) & 
    PowerLogic / ION mérők, \newline Smart megszakítók & 
    S7-1500 PLC, \newline Sentron PAC mérők \\ 
    \hline
    
    \textbf{Kommunikáció} & 
    Wi-Fi (IEEE 802.11) \newline REST API / JSON & 
    Zárt ipari hálózat \newline Modbus/TCP & 
    Ipari Ethernet \newline PROFINET \\ 
    \hline
    
    \textbf{Adatbázis} & 
    Prometheus \newline (Idősoros DB) & 
    MS SQL Server \newline (Relációs DB) & 
    WinCC Archívum \newline (Integrált) \\ 
    \hline
    
    \textbf{Vizualizáció} & 
    Grafana \newline (Webes Dashboard) & 
    Power Monitoring Expert \newline (Web kliens) & 
    WinCC Professional \newline (HMI / SCADA) \\ 
    \hline
    
    % --- FUNKCIONALITÁS ÉS ÜZLET ---
    \multicolumn{4}{|c|}{\textit{Funkcionalitás és Költségek}} \\
    \hline
    
    \textbf{Analitika} & 
    Alapvető mérés \newline + Water-filling algoritmus & 
    Energiaminőség (EN 50160), \newline Zavarelemzés & 
    Terhelésmenedzsment, \newline ISO 50001 riportok \\ 
    \hline
    
    \textbf{Bekerülési költség} & 
    \textbf{Alacsony} \newline (< 1 000 €) & 
    \textbf{Magas} \newline (Licenc + Eszközök) & 
    \textbf{Magas} \newline (Licenc + PLC hardver) \\ 
    \hline
    
    \textbf{Támogatás} & 
    Közösségi (garancia nélkül) & 
    Gyártói 24/7 Support, \newline Hivatalos tanúsítvány & 
    Gyártói Support, \newline TÜV / Szabványi garancia \\ 
    \hline
    
    \end{tabularx}
    
    \caption{A tervezett rendszer összehasonlítása az ipari sztenderdekkel}
    \label{tab:rendszer_osszehasonlitas}
\end{table}