\chapter{Komponensek megvalósítása}

\section{Eszközök}

\subsection{Végpontok}

\subsubsection{Autótöltő}

\subsubsection{Megszakító}

\subsection{Kontroll szerver}

\subsection{Adatbázis}

A rendszer által generált adatok tárolásához egy Prometheus adatbázist használok. 
A Prometheus egy nyílt forráskódú idősoros adatbázis, ami inkább felhő környezetben ismert, 
de ugyanolyan hasznos az IoT-telemetria számára. Minden adatot időbélyegzett értéksorozatként kezel.
Ezeket lehet tárolni és lekérdezni.
\cite{electrofunsmart:iotserver}
\cite{prometheus:dimenzionális}

Esetemben minden metrika tárhelyeként szolgál. Ez lehetővé teszi, 
hogy megőrizzem a töltési áramok történetét és ez alapján irányítsam a rendszert.

\begin{figure}[!ht]
    \centering
    \includegraphics[width=0.5\textwidth, keepaspectratio]{figures/maxresdefault.jpg}
    \caption{Prometheus \cite{youtube:someid}} 
\end{figure}

A Flask szerver-ből könnyű továbbítani az adatokat. A megközelítés amit én használtam hogy egy HTTPS /metrics 
végpont elérhetővé tettem. Amin prometheus által olvasható formában hirdettem az adatokat. 
Például a Flask alkalmazás tudja továbbítani a mért számokat:
\begin{lstlisting}
    current_gauge = prometheus_client.Gauge('ev_charger_current', 'Current draw of EV charger', ['charger']). 
\end{lstlisting}

Ha olvasás érkezik, a szerver frissíti a számokat (egyébként ezt periodikusan is megteszi)
\begin{lstlisting}
    current_gauge.labels(charger=id).set(value). 
\end{lstlisting}

A Prometheus-nak előre megkell adni az ip-címeket a konfugorációs filejában
(a scrape konfigurációján keresztül), hogy időszakonként megnézze a Flask szerver 
/metrics URL-jét. 
Ez azért előnyösebb mert utólag ezeket már nem lehet állítani a prometheusban indítás után.
A szerver, pedig egy stabil IP címen van. A sok fizikai végpontról, pedig a szerver gyújt ahol elértem, hogy 
üzem közben is lehessen új végpontokat hozzáadni vagy módosítani.

Amikor a Prometheus olvas, a Flask az összes aktuális értéket szöveges 
Prometheus metrika formátumban adja ki. A Prometheus ezután ezeket az értékeket a metrikanévvel és címkékkel 
indexelve tárolja. Ez a lehívás alapú felügyelet jól illeszkedik a Prometheus működéséhez. 
A Prometheus adatai megjeleníthetők a Grafana által is és összetett lekérdezések írhatók 
például a teljes áram kiszámítására, amihez szükségem is volt nekem rendszer irányítsásához.

\subsubsection{Prometheus adatgyűjtés kezelése}

A mikrokontroller több metrikát is mér, 
amit belső változókba elment. Jelenleg teszt célokból ezek, csak kézzel megadott számok.

\begin{lstlisting}[language=HTML]
    {
      "# HELP": "esp8266_current Current sensor reading.",
      "# TYPE": "esp8266_current gauge",
      "esp8266_current0": 1.20,
      "esp8266_current1": 2.50
    }
\end{lstlisting}

Ez a formátum megengedi, hogy ezt a /metrics endpointon a prometheus folyamatosan lekérdezze a mikrokontrollerektől. \\

A formátumot a következő függvény hozza létre és küldi:

\begin{lstlisting}[language=C]
    sendMetricsToEndpoint()
    ...
    server.send(200, "text/plain", metrics);
\end{lstlisting}

\subsubsection{Prometheus lekérdezések kezelése}

\begin{lstlisting}[language=C]
    queryPrometheus()
\end{lstlisting}

Ez a függvény egy HTTP GET kérést küld a Prometheus szervernek, amely a esp8266\_total\_current 
metrikát kérdezi le és a prometheusValue változóba írja be.

\begin{lstlisting}
    /api/v1/query?query=esp8266_total_current
\end{lstlisting}

A fentebbi endpointon. \\
A lekérdezés sikerességét a httpCode ellenőrzésével teszem amennyiben ez 200-at ad vissza az értéket eltárolom 
és kiírom a soros kommunikáción ellenőrzés céljából.

\begin{lstlisting}[language=C]
    if (httpCode == HTTP_CODE_OK) {
      String payload = http.getString();
      Serial.println("Response from Prometheus:");
      Serial.println(payload);  \texttt{Adat JSON-be nyomtatása}

      DynamicJsonDocument doc(1024);
      DeserializationError error = deserializeJson(doc, payload);

      if (error) {
        Serial.print(F("JSON deserialization failed: "));
        Serial.println(error.c_str());
        return;
      }
    }
\end{lstlisting}

Mivel a lekérdezés egy JSON formátumú változót ad vissza és ennek feldolgozása nehézkes ezért ezt rögtön szám 
formátumba alakítom későbbi feldolgozás céljából.

\begin{lstlisting}[language=C]
    
    const char* status = doc["status"];
    if (String(status) == "success") {
      
      const char* valueStr = doc["data"]["result"][0]["value"][1];
      
      prometheusValue = String(valueStr).toFloat();

      Serial.print("Extracted Prometheus Value: ");
      Serial.println(prometheusValue);
    }
\end{lstlisting}

A fenti rész kinyeri az adatot JSON formátumból és szám formátumba írja.\\
\\
Természetesen az egész queryPrometheus loop-ban ismétlődve fut, hogy a kontroller folyamatosan frissítse az értékeket. 
Jelenleg a gyakoriságot 30 másodpercre állítottam, hogy ne terhelje a próbák során feleslegesen a hálózatot, 
de gyorsabb válaszidő érdekében ez növelhető.
\subsection{Megjelenítés}