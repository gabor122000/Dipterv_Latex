\chapter{Komponensek megvalósítása}

\section{Eszközök}

\subsection{Végpontok}

\subsubsection{Autótöltő}

\subsubsection{Megszakító}

\subsection{Kontroll szerver}

\subsection{Adatbázis}

A rendszer által generált adatok tárolásához egy Prometheus adatbázist használok. 
A Prometheus egy nyílt forráskódú idősoros adatbázis, ami inkább felhő környezetben ismert, 
de ugyanolyan hasznos az IoT-telemetria számára. Minden adatot időbélyegzett értéksorozatként kezel.
Ezeket lehet tárolni és lekérdezni.
\cite{electrofunsmart:iotserver}
\cite{prometheus:dimenzionális}

Esetemben minden metrika tárhelyeként szolgál. Ez lehetővé teszi, 
hogy megőrizzem a töltési áramok történetét és ez alapján irányítsam a rendszert.

\begin{figure}[!ht]
    \centering
    \includegraphics[width=0.5\textwidth, keepaspectratio]{figures/maxresdefault.jpg}
    \caption{Prometheus \cite{youtube:someid}} 
\end{figure}

A Flask szerver-ből könnyű továbbítani az adatokat. A megközelítés amit én használtam hogy egy HTTPS /metrics 
végpont elérhetővé tettem. Amin prometheus által olvasható formában hirdettem az adatokat. 
Például a Flask alkalmazás tudja továbbítani a mért számokat:
\begin{lstlisting}
    current_gauge = prometheus_client.Gauge('ev_charger_current', 'Current draw of EV charger', ['charger']). 
\end{lstlisting}

Ha olvasás érkezik, a szerver frissíti a számokat (egyébként ezt periodikusan is megteszi)
\begin{lstlisting}
    current_gauge.labels(charger=id).set(value). 
\end{lstlisting}

A Prometheus-nak előre megkell adni az ip-címeket a konfugorációs filejában
(a scrape konfigurációján keresztül), hogy időszakonként megnézze a Flask szerver 
/metrics URL-jét. 
Ez azért előnyösebb mert utólag ezeket már nem lehet állítani a prometheusban indítás után.
A szerver, pedig egy stabil IP címen van. A sok fizikai végpontról, pedig a szerver gyújt ahol elértem, hogy 
üzem közben is lehessen új végpontokat hozzáadni vagy módosítani.

Amikor a Prometheus olvas, a Flask az összes aktuális értéket szöveges 
Prometheus metrika formátumban adja ki. A Prometheus ezután ezeket az értékeket a metrikanévvel és címkékkel 
indexelve tárolja. Ez a lehívás alapú felügyelet jól illeszkedik a Prometheus működéséhez. 
A Prometheus adatai megjeleníthetők a Grafana által is és összetett lekérdezések írhatók 
például a teljes áram kiszámítására, amihez szükségem is volt nekem rendszer irányítsásához.


\subsection{Megjelenítés}