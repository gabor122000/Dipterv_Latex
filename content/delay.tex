% =====================================================================
\chapter{A vezérlési késleltetés és a rendszer-reakcióidő elemzése}
% =====================================================================

\section{Bevezetés: A reakcióidő kritikus szerepe}

A dolgozatban bemutatott keretrendszer nem csupán egy statikus 
felügyeleti eszköz, hanem egy aktív, zárt hurkú vezérlőrendszer 
(closed-loop control system). Míg a 9. fejezet (Rendszertesztek) 
a rendszer \emph{funkcionális} helyességét igazolta -- azaz, helyes 
kiosztást eredményez --, ez a fejezet a rendszer \emph{nem-funkcionális}, 
időbeli viselkedését, specifikusan a késleltetését (latency) vizsgálja.

Energetikai rendszerek felügyeleténél a reakcióidő kritikus tényező. 
A keretrendszer alapvető célja, hogy a szoftveres, intelligens 
allokációval (\texttt{ALLOC\_MAX\_TOTAL}) megelőzze a fizikai, 
"nyers" védelmi eszközök (megszakítók, \texttt{BREAKER\_MAX\_TOTAL} 
leoldását. A rendszer gyakorlatilag egy 
"versenyt fut" az idővel: a szoftveres vezérlőhuroknak gyorsabban 
kell észlelnie és kezelnie a túlterhelést, mint ahogy a fizikai 
megszakító áramkörileg reagálna rá.

Ha a vezérlőhurok késleltetése túl magas, a rendszer -- bár 
funkcionálisan helyes -- a gyakorlatban mégis kudarcot vall, 
mivel a túlterhelés hatására a megszakítók leoldanak, mielőtt 
a szoftveres korlátozás (\texttt{cap}) érvénybe léphetne. 
Ez a fejezet elemzi a késleltetés forrásait és levonja a rendszer 
beállítására vonatkozó legfontosabb következtetést.

\section{A vezérlési hurok késleltetési komponensei}
\label{sec:latency-components}

Ahhoz, hogy a teljes rendszer-reakcióidőt (\(T_{total}\)) megértsük, 
fel kell bontanunk a teljes vezérlési hurkot az egyes komponensek 
által hozzáadott késleltetésekre. Egy túlterheléses esemény észlelése 
és kezelése a következő lépésekből áll:

\begin{enumerate}
    \item \textbf{Mérés (\(T_{measure}\)):} Az \texttt{ESP8266} végpont 
    árammérő szenzora (pl. SCT-013) fizikailag megméri az. 
    Ez a lépés közel valós idejű, 
    késleltetése elhanyagolható (ms nagyságrendű).

    \item \textbf{Publikálás (\(T_{publish}\)):} Az \texttt{ESP8266} 
    frissíti a saját belső állapotát és elérhetővé teszi azt 
    a \texttt{/metrics} végponton. Ez szintén gyors, szoftveres művelet.

    \item \textbf{Adatgyűjtés (\(T_{scrape}\)):} A Prometheus szerver, 
    a konfigurációjában (pl. \texttt{scrape\_interval}) meghatározott 
    időközönként lekérdezi a végpont \texttt{/metrics} URL-jét. 
    \emph{Ez a rendszer egyik legjelentősebb késleltetési tényezője.} 
    Egy tipikus 15 másodperces időköz esetén a mérés és annak 
    Prometheus általi észlelése között átlagosan 7.5, 
    legrosszabb esetben 15 másodperc telik el.

    \item \textbf{Adatfeldolgozás (\(T_{query}\)):} A központi 
    Kontroll Szerver lekérdezi a Prometheus adatbázisból az 
    összesített és egyedi áramadatokat. 
    Bár a PromQL lekérdezés gyors, ez a lépés 
    is a Prometheus belső adatmodelljének frissítési 
    ciklusától függ.

    \item \textbf{Döntéshozatal (\(T_{algo}\)):} A Python szerver 
    a beolvasott adatok alapján lefuttatja a \texttt{max--min fair} 
    algoritmust. Ahogy a 4. fejezetben láttuk, ennek bonyolultsága 
    \(O(n \log n)\), ami kis (\(n=3..10\)) fogyasztószám esetén 
    gyakorlatilag azonnali (elhanyagolható \(\mu\)s - ms tartomány).

    \item \textbf{Beavatkozás (\(T_{command}\)):} A szerver a kiszámolt 
    új korlátokat (\texttt{cap}) elküldi a végpontoknak, jellemzően 
    Modbus/TCP protokollon keresztül. Ez a lépés 
    magában foglalja a hálózati késleltetést (Wi-Fi, LAN).

    \item \textbf{Végrehajtás (\(T_{execute}\)):} Az \texttt{ESP8266} 
    fogadja a Modbus parancsot és beállítja az autótöltő maximális 
    áramát. Ez is egy gyors, eszköz-oldali művelet.
\end{enumerate}

A teljes reakcióidő tehát: 
\begin{equation}
    T_{total} = T_{measure} + T_{publish} + T_{scrape} + T_{query} 
              + T_{algo} + T_{command} + T_{execute}
\end{equation}

\section{A domináns késleltetési tényezők azonosítása}

A \ref{sec:latency-components} pontban felvázolt láncban egyértelműen 
látszik, hogy a késleltetést nem a számítási kapacitás (az algoritmus) 
limitálja, hanem az adatgyűjtés architektúrája.

A teljes késleltetést (\(T_{total}\)) alapvetően két tényező dominálja:
\begin{enumerate}
    \item \textbf{Prometheus Scrape Interval (\(T_{scrape}\)):} 
    Ez a rendszer "szívverése". A vezérlőhurok nem tud gyorsabban 
    reagálni, mint amilyen sűrűn friss adathoz jut. Ha ez az 
    érték 15 másodperc, a rendszer fizikailag "vakon" repül 
    15 másodpercig, és csak utólag értesül a korábbi eseményekről.
    
    \item \textbf{Hálózati késleltetés (\(T_{command}\)):} 
    A Wi-Fi-alapú kommunikáció (mind a REST API , 
    mind a Modbus/TCP esetén) további, változó hosszúságú (jitter) 
    késleltetést vihet a rendszerbe, különösen terhelt vagy 
    zajos hálózati környezetben.
\end{enumerate}

Míg a hálózati késleltetés általában tíz-száz milliszekundum 
nagyságrendű, a \texttt{scrape\_interval} másodperces nagyságrendű. 
Ebből következik, hogy \textbf{a rendszer reakcióidejét döntő 
mértékben a Prometheus adatgyűjtési gyakorisága határozza meg.}

\section{A védelmi beállítások és a szoftveres vezérlés kapcsolata}

Ez a késleltetés-elemzés közvetlen és kritikus hatással van a rendszer 
védelmi beállításainak konfigurálására. A keretrendszer két védelmi 
szintet valósít meg:

\begin{itemize}
    \item \textbf{1. Szoftveres Védelem (Intelligens Allokáció):} 
    Ezt a \texttt{ALLOC\_MAX\_TOTAL} keret szabályozza. Célja a 
    \emph{komfort} és a \emph{hatékonyság} biztosítása, a keretek 
    betartása és a leoldások aktív \emph{megelőzése}. 
    Reakcióideje lassú (\(T_{total}\)).
    
    \item \textbf{2. Fizikai Védelem (Biztonsági Megszakító):} 
    Ezt a \texttt{BREAKER\_MAX\_TOTAL} küszöb határozza meg. 
    Célja a \emph{biztonság}, a hálózat fizikai védelme. 
    Reakcióideje azonnali (fizikai).
\end{itemize}

A rendszer stabil működésének feltétele, hogy e két szint 
megfelelően legyen összehangolva. A \texttt{BREAKER\_MAX\_TOTAL} 
értékét nem lehet a \texttt{ALLOC\_MAX\_TOTAL} értékéhez közel beállítani.

Szükséges egy \textbf{biztonsági puffer} (headroom) fenntartása 
a két küszöb között. Ennek a puffernek nagyobbnak kell lennie, 
mint amennyi áramnövekedés a rendszerben történhet 
a \emph{teljes szoftveres reakcióidő} (\(T_{total}\)) alatt.

\paragraph{Konfigurációs példa.}
Tegyük fel, hogy:
\begin{itemize}
    \item A szoftveres keret \(\texttt{ALLOC\_MAX\_TOTAL} = 90\,A\).
    \item A Prometheus \texttt{scrape\_interval} és a teljes hurok 
          késleltetése \(\textbf{\(T_{total}\) = 15\,s}\).
    \item A csatlakoztatott eszközök (pl. autótöltők) képesek 
          15 másodperc alatt további 20\,A terhelést bekapcsolni 
          (pl. egy új autó csatlakozik), mielőtt a rendszer reagálna.
\end{itemize}

Ebben az esetben a \texttt{BREAKER\_MAX\_TOTAL} küszöbértéket legalább 
\(90\,A + 20\,A = 110\,A\)-re (vagy még magasabbra) kell állítani.

Ha a fizikai megszakítót túl "szűkre" állítanánk (pl. 95\,A-re), 
a rendszer instabillá válna. Egy hirtelen 20\,A-es terhelésnövekedés 
(90\,A \(\rightarrow\) 110\,A) azonnal leoldaná a 95\,A-es fizikai 
megszakítót, jóval azelőtt, hogy a 15 másodperces késleltetésű 
szoftveres hurok egyáltalán észlelhette volna a problémát és 
beavatkozhatott volna az allokációval.

\section{Következtetések}

A keretrendszer vezérlési sebességét nem a \texttt{max--min fair} 
algoritmus számítási bonyolultsága, hanem a \texttt{pull}-alapú 
adatgyűjtési architektúra (Prometheus) késleltetése határozza meg. 
Ez a szoftveres késleltetés (\(T_{total}\)) a rendszer egyik 
legfontosabb paramétere.

A fizikai védelmi beállításokat (\texttt{BREAKER\_MAX\_TOTAL}) 
kötelezően ennek a szoftveres késleltetésnek a figyelembevételével 
kell konfigurálni. A két védelmi szint között elegendő 
"fejteret" (headroom) kell hagyni, hogy a szoftveres allokációnak 
legyen ideje reagálni a terhelésváltozásokra, mielőtt a fizikai 
védelem szükségtelenül megszakítaná a szolgáltatást. 
A rendszer finomhangolása tehát egy egyensúlyozás a gyors 
adatgyűjtés (alacsony \texttt{scrape\_interval}, magas hálózati terhelés) 
és a nagyobb védelmi puffer (magasabb \texttt{BREAKER\_MAX\_TOTAL}) 
között.