%--------------------------------------------------------------------------------------
% AI Nyilatkozat
%--------------------------------------------------------------------------------------
%----------------------------------------------------------------------------
\section{Nyilatkozat generatív mesterséges intelligencia alkalmazásáról}
%----------------------------------------------------------------------------

\noindent $\Box$ \textbf{Nem használtam} semmilyen generatív MI segédeszközt.

\noindent $\Box$ \textbf{Használtam} generatív MI segédeszközt. Az MI-vel generált tartalmakat ellenőriztem, a generált kimenetek valóságtartalmáról meggyőződtem, az alábbi táblázatban megfelelően jelöltem minden használatot.

\small{\noindent
    \begin{tabular}{|c|c|c|c|}\hline
        \hspace{40mm} & \hspace{30mm} & \hspace{30mm} & \hspace{30mm} \\ 
        \textbf{Felhasználási módok}    & \textbf{Generatív MI}      & \textbf{Érintett részek}  & \textbf{Használat becsült} \\
        & \textbf{eszköz(ök) neve} & (fejezet, oldalszám, & \textbf{aránya} (felhasználási \\ 
        & & hivatkozás) & módonként) \\
        \hspace{40mm} & \hspace{30mm} & \hspace{30mm} & \hspace{30mm} \\ \hline \hline
        Irodalomkutatás & Gemini & 2. fejezet (3-6. o.) & 5\% \\ \hline
        \textbf{Prompt lényegi része}  &\multicolumn{3}{|p{10cm}|}{Open source alternatívák keresése ipari energetikai felügyeleti rendszerekre (Siemens/Schneider) és azok összehasonlítása.} \\ \hline \hline
        Programkód generálása & Gemini & 4-6. fejezet (kódlisták) & 10\% \\ \hline
        \textbf{Prompt lényegi része } & \multicolumn{3}{|p{10cm}|}{Flask REST API alapstruktúra létrehozása, Prometheus export formátum ellenőrzése, Kubernetes deployment YAML fájlok szintaxisának javítása.} \\ \hline \hline
        Új ötletek, megoldási & & & \\ 
        javaslatok generálása & & & \\ \hline
        \textbf{Prompt lényegi része}  &\multicolumn{3}{|l|}{} \\ \hline \hline
        Vázlat létrehozása & & & \\
        (szövegstruktúra, & & & \\ 
        vázlatpontok) & & & \\ \hline
        \textbf{Prompt lényegi része}  &\multicolumn{3}{|l|}{} \\ \hline \hline
        Szövegblokkok létrehozása & Gemini & Abstract (9. o.) & 10\% \\ \hline
        \textbf{Prompt lényegi része}  &\multicolumn{3}{|p{10cm}|}{A magyar nyelvű kivonat angolra fordítása és szakmai nyelvezetének ellenőrzése.} \\ \hline \hline
        Képek generálása & Gemini & 3.2. ábra (\ref{fig:Rendszerarchitektúra}), & 10\% \\
        illusztrációs célból & & 6.2. ábra (\ref{kubernetes topology}) & \\ \hline
        \textbf{Prompt lényegi része}  &\multicolumn{3}{|p{10cm}|}{IoT rendszerarchitektúra és Hibrid Kubernetes topológia diagramjának generálása vizualizációs célból.} \\ \hline \hline
        Adatvizualizáció, & & & \\
        grafikonok generálása & & & \\
        adatpontok alapján & & & \\ \hline
        \textbf{Prompt lényegi része}  &\multicolumn{3}{|l|}{} \\ \hline \hline
        Prezentáció készítése & & & \\ \hline
        \textbf{Prompt lényegi része}  &\multicolumn{3}{|l|}{} \\ \hline \hline
        Egyéb (nevezze meg) & & & \\ \hline
        \textbf{Prompt lényegi része}  &\multicolumn{3}{|l|}{} \\ \hline
        \multicolumn{3}{|l|}{\textbf{Összesített százalékos érték (a feladat érdemi részére nézve):}} & \textbf{9\%} \\ \hline
        \multicolumn{4}{|l|}{\textbf{Összesített érték rövid, szöveges indoklása:}} \\ 
        \multicolumn{4}{|p{13cm}|}{A generatív MI eszközt kiegészítő jelleggel használtam: ismétlődő kódrészletek (boilerplate) generálására, szintaxis ellenőrzésre (Kubernetes YAML), az angol nyelvű összefoglaló javítására, valamint két illusztrációs ábra (architektúra, topológia) elkészítésére. A dolgozat érdemi szöveges részét és a rendszertervezést önállóan végeztem.} \\ 
        \multicolumn{4}{|l|}{} \\
        \multicolumn{4}{|l|}{} \\
        \multicolumn{4}{|l|}{} \\ \hline
        
    \end{tabular}
} % end of \small


